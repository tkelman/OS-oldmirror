\documentclass{article}
\usepackage[colorlinks=true]{hyperref}
\usepackage{html}
\usepackage{lscape}
\usepackage{amsmath,amssymb}
\usepackage{threeparttable}
\usepackage{xtab}
\usepackage{bonmin}


\title{\Bonmin\ Users' Manual}
\author{Pierre Bonami and Jon Lee}


\begin{document}
\maketitle

%%% Local Variables:
%%% mode: latex
%%% TeX-master: "Bonmin_UsersManual"
%%% End:

\html{

\begin{PageSummary}
\PageName{Introduction}
\PageSection{Types of problems solved}{MathBack}
\PageSection{Algorithms}{Algos}
\PageSection{Required third party code}{ThirdP}
\PageSection{Supported platforms}{Support}
\end{PageSummary}


\begin{quickref}
\quickcitation{An algorithmic framework for convex MINLP. Bonami et.al.}{\BetalLink}
\quickcitation{Algorithms and Software for Convex Mixed Integer Nonlinear Programs. Bonami, Kilinc, Linderoth}{\HotMINLPLink}
\quickcitation{An outer-approximation algorithm for a class of MINLP. M. Duran and I.E. Grossmann. Mathematical Programming}{\DGLink}
\quickcitation{Branch and bound experiments in convex nonlinear integer programming. O.K. Gupta and V. Ravindran.}{\GuptaLink}
\quickcitation{Solving MINLP by outer approximation.
  R. Fletcher and S. Leyffer. Mathematical Programming.}{\FLLink}
\quickcitation{An LP/NLP based branched and bound algorithm for convex MINLP optimization problems. I. Quesada and I.E. Grossmann.
   Computers and Chemical Engineering.}{\QGLink}\end{quickref}
}

\PageTitle{\latexhtml{Introduction}{\Bonmin}}{sec:Intro}
\Bonmin\ (Basic Open-source Nonlinear Mixed INteger programming)
is an open-source code for solving general MINLP (Mixed
Integer NonLinear Programming) problems.
 It is distributed on
\COINOR
\latexhtml{{(\tt www.coin-or.org)}}{}
under the CPL (Common Public
License). The CPL is a license approved by the
\footlink{http://www.opensource.org}{OSI},
(Open Source Initiative),
 thus \Bonmin\ is OSI
Certified Open Source Software.

There are several algorithmic choices that can be selected with \Bonmin.
{\tt B-BB} is a NLP-based branch-and-bound algorithm,
{\tt B-OA} is an
outer-ap\-prox\-i\-ma\-tion decomposition algorithm, {\tt B-iFP} is an iterated
feasibility pump algorithm, {\tt B-QG} is an
implementation of  Quesada and Grossmann's branch-and-cut algorithm,
{\tt B-Hyb} is a hybrid outer-ap\-prox\-i\-ma\-tion based
branch-and-cut algorithm and {\tt B-Ecp} is a variant of {\tt B-QG} based
on adding additional ECP cuts.


Some of the algorithmic choices require the ability to solve MILP
(Mixed Integer Linear Programming) problems and NLP (NonLinear
Programming) problems. The default solvers for these are,
respectively, the COIN-OR codes \Cbc\ and \Ipopt. In turn,
{\tt Cbc} uses further COIN-OR modules: \Clp\ (for LP (Linear
Programming) problems), \Cgl\ (for generating MILP cutting
planes), as well as various other utilities. It is also possible to
step outside the open-source realm and use
\Cplex\ as the MILP solver and FilterSQP as the NLP solver. 

Additional documentation can be found on the {\tt Bonmin}
\latex{ homepage at 
$$
         \hbox{\tt http://www.coin-or.org/Bonmin}
$$
 and wiki at 
$$
         \hbox{\tt https://projects.coin-or.org/Bonmin}
$$
}

\html{
\href{http://www.coin-or.org/Bonmin}{homepage} 
and \href{https://projects.coin-or.org/Bonmin}{wiki}.}

\subsectionHs{Types of problems solved}{MathBack}
\Bonmin\ solves MINLPs of the form

%\left\{
\begin{align*}
%\begin{array}{l}
&\min f(x) \\
& {\rm s.t.} \\
&g^L \leq g(x) \leq g^U,\\
& x^L \leq x \leq x^U, \\
&x \in \mathbb{R}^n, \;  x_i \in \mathbb{Z} \; \forall i \in I,
%\end{array}
\end{align*}
%\right.
where the functions $f :~\{x\in \mathbb{R}^n : x^L \leq x \leq x^U
\}~ \rightarrow~\mathbb{R}$ and $g:~\{x\in \mathbb{R}^n : x^L \leq x
\leq x^U \}~\rightarrow~\mathbb{R}^m$ are assumed to be twice
continuously differentiable, and $I \subseteq \{1, \ldots,n \}$. We
emphasize that \Bonmin\ treats problems that are cast
in {\em minimization} form.

The different methods that \Bonmin\ implements are exact algorithms when the functions $f$ and $g$ are
convex but are only heuristics when this is not the case (i.e., \Bonmin\ is not a \emph{global} optimizer).\\

\subsectionHs{Algorithms}{Algos}
\Bonmin\ implements six different algorithms for solving
MINLPs:

\begin{itemize}
\item {\tt B-BB}: a simple branch-and-bound algorithm based on solving
a continuous nonlinear program at each node of the search tree and
branching on variables \mycite{Gupta80Nonlinear}{Gupta 1980};
we also allow the possibility of SOS (Type 1) branching
\item {\tt B-OA}: an outer-approximation based decomposition algorithm
\latexhtml{\cite{DG,FL}}
{[\href{\DGLink}{Duran Grossmann 1986},\href{\FLLink}{Fletcher Leyffer 1994}]}
\item {\tt B-QG}: an outer-approximation based branch-and-cut
algorithm
\citeH{QG}{\QGLink}{Quesada Grossmann 1994}
\item {\tt B-Hyb}: a hybrid outer-approximation/nonlinear programming
based branch-and-cut algorithm \citeH{Betal}
{\BetalLink}{Bonami et al. 2008}
\item {\tt B-Ecp}: another outer-approximation based branch-and-cut inspired
by the settings described in \citeH{abhishek.leyffer.linderoth:06}{\AbhishekLink}{Abhishek Leyffer Linderoth 2006}
\item {\tt B-iFP}: an iterated feasibility pump algorithm \citeH{bonami.etal:06}{\FPLink}{Bonami Cornu\eacute jols Lodi Margot 2009}.
\end{itemize}

In this manual,  we will not go into a further description of these algorithms.
Mathematical details of these algorithms 
and some details of their implementations can be found in \citeH{Betal}{\BetalLink}{Bonami et al. 2008} and \citeH{hot:2009}{\HotMINLPLink}{Bonami K\i ln\i c Linderoth 2009}.

Whether or not you are interested in the details of the algorithms, you certainly
want to know which one of these six algorithms you should choose to solve
your particular problem.
For convex MINLPs, experiments we have made on a reasonably large test set of problems point in favor of using {\tt B-Hyb}
(it solved the most of the problems in our test set in 3 hours of computing time).
Nevertheless, there are cases where {\tt B-OA} is much faster than {\tt B-Hyb} and others where {\tt B-BB} is interesting.
{\tt B-QG} and {\tt B-ECP} correspond mainly to a specific parameter setting of {\tt B-Hyb} but they can be faster in some case. {\tt B-iFP} is more tailored at finding quickly good solutions to very hard convex MINLP.
For nonconvex MINLPs, we strongly recommend using {\tt B-BB} (the outer-approximation algorithms
have not been tailored to treat nonconvex problems at this point). Although even {\tt B-BB} is only a
heuristic for such problems, we have added several
options to try and improve the quality of the solutions it provides (see \latexhtml{Section \ref{sec:non_convex}}{\href{\OptSetPage \#sec:non_convex}{non convex options}}).
Because it is appliable to more classes problem {\tt B-BB} is the default algorithm in \Bonmin.

\subsectionHs{Required third party code}{ThirdP}
In order to run {\Bonmin}, you have to download other external
libraries (and pay attention to their licenses!):
\begin{itemize}
\item \href{\LapackAddr}{Lapack} (Linear Algebra
PACKage)
\item \href{\BlasAddr}{Blas} (Basic Linear Algebra
Subroutines)
\item a sparse linear solver that is supported by Ipopt, e.g., MA27 from the
\href{\AslAddr}{HSL}
(Harwell Subroutine Library), MUMPS, or Pardiso.
\end{itemize}

Note that Lapack and the Blas are free for commercial use from the
\footlink{http://www.netlib.org}{Netlib Repository}, but they are
not OSI Certified Open Source Software. The linear solver MA27 is
freely available for noncommercial use.

The above software is sufficient to run \Bonmin\ as a
stand-alone C++ code, but it does not provide a modeling language.
For functionality from a modeling language, \Bonmin\ can be
invoked from \footlink{http://www.ampl.com}{\tt Ampl} (no extra installation is required provided
that you have a licensed copy of {\tt Ampl} installed), though you
need the {\tt ASL} (Ampl Solver Library) which is obtainable from the Netlib.

\Bonmin\ can use FilterSQP \citeH{FilTer}{\FilterLink}{FletcherLeyffer1998} as an alternative to \Ipopt\ for solving NLPs.

Also, in the outer approximation methods {\tt B-OA} and {\tt B-iFP}, some MILP problems are
solved. By default \Bonmin\ uses  \Cbc\ to solve them, but it can also be set up to use
the commercial solver \footlink{http://www.ilog.com/products/cplex/product/mip.cfm}{\Cplex}.

\subsectionHs{Tested platforms}{Support}
\Bonmin\ has been installed on the following systems:
\begin{itemize}
\item Linux using g++ version 3.* and 4.* until 4.3 and Intel 9.* and 10.*
\item Windows using version Cygwin 1.5.18
\item Mac OS X using gcc 3.* and 4.* until 4.3 and Intel 9.* and 10.*
\item SunOS 5 using gcc 4.3
\end{itemize}



\begin{PageSummary}
\PageName{Downloading \Bonmin}
\PageSection{Obtaining \Bonmin}{sec:obtain}
\PageSection{Obtaining required third party code}{sec:obtain_3rd}
\end{PageSummary}

\begin{quickref}
\quickcitation{Bonmin Wiki Pages}{\linkCoin Bonmin}
\quickcitation{subversion web page}{http://subversion.tigris.org/}
\quickcitation{Using subversion on windows}{http://www.coin-or.org/faqs.html\#q4}
\quickcitation{Linear Algebra PACKage}{http://www.netlib.org/lapack/}
\quickcitation{Basic Linear Algebra Subroutines}{http://www.netlib.org/blas/}
\quickcitation{Harwell Subroutine Library}{http://www.cse.clrc.ac.uk/nag/hsl/contents.shtml}
\quickcitation{Ampl Solver Library}{http://www.ampl.com}
\end{quickref}

\PageTitle{Obtaining \Bonmin}{sec:obtain}



The \Bonmin\ package consists of the source code for the \Bonmin\
project but also source code from other \COINOR\ projects:
\begin{itemize}
\item \BuildTools
\item \Cbc
\item \Cgl
\item \Clp
\item \CoinUtils
\item \Ipopt
\item \Osi
\end{itemize}

When downloading the \Bonmin\ package you will download the source code for all these and
libraries of problems to test the codes.\\

Before downloading \Bonmin\ you need to know which branch of Bonmin you want to download. 
In particular you need to know if you want to download the latest version from:
\begin{itemize}
    \item the Stable branch, or from
    \item the Released branch. 
\end{itemize}
These different version are made according to the guidelines of COIN-OR. The interpretation of these guidelines for the Bonmin project is explained on the wiki pages of Bonmin.

The main distinction between the Stable and Release branch is that a stable version that we propose to download may evolve over time to include bug fixes while a released version will never change. The released versions present an advantage in particular if you want to make experiments which you want to be able to reproduce the stable version presents the advantage that it is less work for you to update in the event where we fix a bug.

The easiest way to obtain the released version is by downloading a compressed archive from \link{http://www.coin-or.org/Tarballs/Bonmin/}{Bonmin archive directory}. The latest release is Bonmin-0.99.0.

The only way to obain one of the stable versions is through \link{http://subversion.tigris.org/}{subversion}.

In Unix\footnote{UNIX is a registered trademark of The Open
Group.}-like environments, to download the latest stable version of Bonmin (0.99) in a sub-directory, say {\tt Bonmin-0.99} 
issue the following command
%\begin{verbatim}
\break

\begin{colorverb}
 \noindent {\tt  svn co
https://projects.coin-or.org/svn/Bonmin/stable/0.99 Bonmin-0.99 }
\end{colorverb}

\noindent This copies all the necessary COIN-OR files to compile \Bonmin\ to
{\tt Bonmin-0.99}. To download \Bonmin\ using svn on Windows,
follow the instructions provided at
\link{http://www.coin-or.org/faqs.html\#q4}{COIN-OR}.

\subsectionH{Obtaining required third party code}{sec:obtain_3rd}
\Bonmin\ needs a few external packages which are not included in the \Bonmin\ package:
\begin{itemize}
\item Lapack (Linear Algebra PACKage)
\item Blas (Basic Linear Algebra Subroutines)
\item the sparse linear solver MA27 from the Harwell Subroutine Library and optionally (but strongly recommended) MC19 to enable automatic scaling in \Ipopt.
\item optionally ASL (the Ampl Solver Library), to be able to use \Bonmin\ from Ampl.
\end{itemize}

Since these third-party software modules are released under licenses
that are incompatible with the CPL, they cannot be included for
distribution with \Bonmin\ from COIN-OR, but you will find scripts
to help you download them in the subdirectory {\tt ThirdParty} of
the \Bonmin\ distribution\footnote{In most Linux distribution and
CYGWIN, Lapack and Blas are available as prebuilt binary packages in
the distribution (and are probably already installed on your
machine).}. For details on how to obtain these package, refer to the
instructions in
\link{http://www.coin-or.org/Ipopt/documentation/node13.html}{Section 2.2} of the Ipopt manual.\\


\begin{PageSummary}
\PageName{Installation}
\PageSection{Installing \Bonmin}{sec:install}
\PageSection{Configuring {\tt Cplex}}{sec:cplex_install}
\PageSection{Compiling \Bonmin\ in a external directory}{sec:vpath}
\PageSection{Building documentation}{sec:ref_man}
\PageSection{Running test program}{sec:test}
\end{PageSummary}

\begin{quickref}
\quickcitation{Generic Coin Installation on Coin BuildTools web page}{\linkCoin BuildTools}
\quickcitation{Known issues for building Coin packages}{\linkCoin BuildTools}
\quickcitation{\Bonmin\ build Wiki page.}{\linkCoin Bonmin/wiki/GettingStarted}
\quickcitation{Specific Instructions for building \Bonmin\ on Cygwin}{\linkCoin Bonmin/Wiki/CygwinInstall}
\quickcitation{Specific instructions for building \Bonmin\ on Mac OSX}{\linkCoin Bonmin/Wiki/OsXInstall}
\end{quickref}
\PageTitle{Installing \Bonmin}{sec:install}
The build process for \Bonmin\ should be fairly automatic as it uses
\href{http://sources.redhat.com/autobook/autobook/}{GNU autotools}.
  It has been successfully compiled and run on the following platforms:
\begin{itemize}
\item Linux using g++ version 4.5
\item Windows using version Cygwin 1.5.18
\item Mac OS X using gcc 4.5
\end{itemize}

For Cygwin and OS X some specific setup has to be done prior to instalation. These step are described on the wiki pages of {\tt Bonmin}  \footlink{https://projects.coin-or.org/Bonmin/wiki/CygwinInstall}{CygwinInstall} and \footlink{https://projects.coin-or.org/Bonmin/wiki/OsxInstall}{OsxInstall}.


\Bonmin\ is compiled and installed using the commands:
\begin{colorverb}
\begin{verbatim}

./configure -C
make
make install

\end{verbatim}
\end{colorverb}

This installs the executable {\tt bonmin} in {\tt Bonmin-\stableVersion/bin}. In what follows, we assume
that you have put the executable {\tt bonmin} on your path.

The {\tt configure} script attempts to find all of the machine specific settings (compiler, libraries,...)
necessary to compile and run the code. Although {\tt configure} should find most of the standard
ones, you may have to manually specify a few of the settings.
The options for the configure script can be found by issuing the command

\begin{colorverb}
\begin{verbatim}

./configure --help

\end{verbatim}
\end{colorverb}

For a more in depth description of these options,
the reader is invited to refer to the COIN-OR {\tt BuildTools} \footlink{\linkCoin BuildTools}{trac page}.

\subsectionH{Specifying the location of {\tt Cplex} libraries}{sec:cplex_install}
If you have {\tt Cplex} installed on your machine, you may want to use it
as the Mixed Integer Linear Programming subsolver in {\tt B-OA}, {\tt B-Hyb}
and {\tt B-iFP}.
To do so you have to specify the location of the header files and libraries.
You can either specify the location of the header files directory by passing it as an
argument to the configure script or by writing it into a {\tt config.site}.\\

In the former case, specify the location of the {\tt Cplex} header files by using the
argument {\tt --with-cplexincdir} and the location of the
{\tt Cplex } library with {\tt --with-cplexlib} (note that on the Linux platform you will also
need to add {\tt -lpthread -lm} as an argument to {\tt --with-cplexlib}).\\

For example, on a Linux machine if {\tt Cplex} is installed in {\tt /usr/ilog}~, you would
invoke configure with the arguments as follows:

\begin{colorverb}
\begin{verbatim}

./configure --with-cplex-incdir=/usr/ilog/cplex/include/ilcplex \
  --with-cplex-lib="-L/usr/ilog/cplex/lib -lcplex -lpthread -lm"
 \end{verbatim}
 \end{colorverb}
 
In the latter case, put a file called {\tt config.site} in a subdirectory named
{\tt share} of the installation directory (if you do not specify an alternate
installation directory to the {\tt configure} script with the {\tt --prefix}
argument, the installation directory is the directory where you execute the
{\tt configure} script). To specify the location of {\tt Cplex}~, insert the
following lines in the {\tt config.site} file:

\begin{colorverb}
 \begin{verbatim}

 with_cplex_lib="-L/usr/ilog/cplex/lib/ -lcplex -lpthread -lm"
 with_cplex_incdir="/usr/ilog/cplex/include/ilcplex"
 
 \end{verbatim}
\end{colorverb}

 (You will find a {\tt config.site} example in the subdirectory {\tt BuildTools} of {\tt coin-Bonmin}.)

\subsectionH{Compiling \Bonmin\ in a external directory}{sec:vpath}
It is recommended to compile \Bonmin\ in a directory different from the source directory ({\tt Bonmin-\stableVersion} in our case).
This is convenient if you want to have several executables compiled for different architectures or have several executables compiled with different options
(debugging and production, shared and static libraries) but also because
you don't modify the directory where the sources are.\\

To do this just create a new directory, for example {\tt Bonmin-build} in the parent directory of
{\tt coin-Bonmin} and run the configure command from {\tt Bonmin-build}:


\begin{colorverb}
\begin{verbatim}

../Bonmin-\stableVersion/configure -C

\end{verbatim}
\end{colorverb}

This will create the makefiles in {\tt Bonmin-build}, and
you can then compile with the usual {\tt make} and {\tt make install}
(in {\tt Bonmin-build}).

\subsectionH{Building the documentation}{sec:ref_man}
The documentation for \Bonmin\ consists of a users' manual (this document) and a reference manual.
You can build a local copy of the reference manual provided that you have Latex
and Doxygen installed on your machine. Issue the command {\tt make
doxydoc} in {\tt coin-Bonmin}. It calls Doxygen to build a copy of the
reference manual. An html version of the reference manual can then
be accessed in {\tt doc/html/index.html}.

%You can also build a pdf
%version of the reference manual by issuing the command {\tt make
%refman.pdf} ({\tt refman.pdf} is placed in the {\tt doc} subdirectory).

\subsectionH{Running the test programs}{sec:test}
By issuing the command {\tt make test}~, you build and run the automatic test program for \Bonmin.



\begin{PageSummary}
\PageName{Running \Bonmin}
\PageSection{On an .nl file}{sec:run_nl}
\PageSection{From Ampl}{sec:run_ampl}
\PageSection{priorities and SOS in Ampl}{sub_sec:prio}
\PageSection{From GAMS}{sec:run_gams}
\PageSection{From a C++ Programm}{sec:run_cpp}
\end{PageSummary}

\begin{quickref}
\quickcitation{Writing \texttt{\bf .nl} files. D.M.~Gay.}{\BibPage \#Gay}
\quickcitation{AMPL: A Modeling Language for Mathematical
Programming, Second Edition, Duxbury Press Brooks Cole Publishing Co., 2003. R.~Fourer and D.M.~Gay and B.W.~Kernighan.}{\BibPage \#AMPL}
\end{quickref}


\PageTitle{Running \Bonmin}{sec:run}
\Bonmin\ can be run
\begin{itemize}
\item [(i)] from a command line on a {\tt .nl} file
(see \mycite{Gay}{Gay2005}),
\item [(ii)] from the modeling language \footlink{http://www.ampl.com}{\tt Ampl} (see
\mycite{AMPL}{Fourrer2003}),
\item[(iii)] from the \footlink{http://www.gams.com/}{Gams} modeling language,
\item [(iv)] by invoking it from a C/C++ program.
\item[(v)] remotely through the \footlink{http://neos.mcs.anl.gov/neos}{NEOS} web interface.
\end{itemize}

In \latexhtml{the subsections that follow}{this page}, we give some details about the
various ways to run \Bonmin.

\subsectionH{On a {\tt .nl} file}{sec:run_nl}
\Bonmin\ can read a {\tt .nl} file which could be generated by {\tt
Ampl} (for example {\tt mytoy.nl} in the {\tt
Bonmin-dist/Bonmin/test} subdirectory). The command line takes just
one argument which is the name of the {\tt .nl} file to be
processed.

For example, if you want to solve {\tt mytoy.nl}, from the {\tt
Bonmin-dist} directory, issue the command:

\begin{colorverb}
\begin{verbatim}

bonmin test/mytoy.nl

\end{verbatim}
\end{colorverb}

\subsectionH{From {\tt Ampl}}{sec:run_ampl}
To use \Bonmin\ from {\tt Ampl} you just need to have the directory where the
{\tt bonmin} executable is in your {\tt \$PATH} and to issue the
command

\begin{colorverb}
\begin{verbatim}

option solver bonmin;

\end{verbatim}
\end{colorverb}

in the {\tt Ampl} environment. Then the next {\tt solve} will
use \Bonmin\ to solve the model loaded in {\tt Ampl}.
After the optimization is finished, the values of the variables in the best-known
or optimal solution can be accessed in {\tt Ampl}. If the optimization is interrupted
with {\tt <CTRL-C>} the best known solution is accessible (this feature is not available in Cygwin).\\

\subsubsectionH{Example {\tt Ampl} model}
A simple {\tt Ampl} example model follows:

\begin{colorverb}
\begin{verbatim}

   # An Ampl version of toy

   reset;

   var x binary;
   var z integer >= 0 <= 5;
   var y{1..2} >=0;
   minimize cost:
       - x - y[1] - y[2] ;

   subject to
       c1: ( y[1] - 1/2 )^2 + (y[2] - 1/2)^2 <= 1/4 ;
       c2: x - y[1] <= 0 ;
       c3: x + y[2] + z <= 2;

   option solver bonmin; # Choose BONMIN as the solver (assuming that
                         # bonmin is in your PATH

   solve;                # Solve the model
   display x;
   display y;

\end{verbatim}
\end{colorverb}

(This example can be found in the subdirectory {\tt Bonmin/examples/amplExamples/} of
the \Bonmin\ package.)

\subsubsectionH{Setting up branching priorities, directions and declaring SOS1 constraints in ampl}{sub_sec:prio}
Branching priorities, branching directions and pseudo-costs can be passed using {\tt Ampl} suffixes.
The suffix for branching priorities is {\tt "priority"} (variables with a higher priority
will be chosen first for branching),
for branching direction is {\tt "direction"} (if direction is $1$ the $\geq$ branch
is explored first, if direction is $-1$ the $\leq$ branch is explored first), for up
and down pseudo costs {\tt "upPseudoCost"} and {\tt "downPseudoCost"} respectively
(note that if only one of the up and down pseudo-costs is set in the {\tt Ampl} model it will
be used for both up and down).\\

For example, to give branching priorities of $10$ to variables {\tt y} and 1 to variable {\tt x}
and to set the branching directions to explore the upper branch first for all variables
in the simple example given, we add before the call to solve:
\begin{colorverb}
\begin{verbatim}

suffix priority IN, integer, >=0, <= 9999;
y[1].priority := 10;
y[2].priority := 10;
x.priority := 1;

suffix direction IN, integer, >=-1, <=1;
y[1].direction := 1;
y[2].direction := 1;
x.direction := 1;

\end{verbatim}
\end{colorverb}

SOS Type-1 branching is also available in \Bonmin\ from {\tt Ampl}. We
follow the conventional way of doing this with suffixes.
Two type of suffixes should be declared:

\begin{colorverb}
\begin{verbatim}
suffix sosno IN, integer, >=1;  # Note that the solver assumes that these
                                #   values are positive for SOS Type 1
suffix ref IN;
\end{verbatim}
\end{colorverb}

Next, suppose that we wish to have variables

\begin{colorverb}
\begin{verbatim}
var X {i in 1..M, j in 1..N} binary;
\end{verbatim}
\end{colorverb}
and the ``convexity'' constraints:

\begin{colorverb}
\begin{verbatim}
subject to Convexity {i in 1..M}:
   sum {j in 1..N} X[i,j] = 1;
\end{verbatim}
\end{colorverb}

(note that we must explicitly include the convexity constraints in the {\tt Ampl} model).

Then after reading in the data, we set the suffix values:
\begin{colorverb}
\begin{verbatim}

# The numbers `val[i,j]' are chosen typically as
#     the values `represented' by the discrete choices.
let {i in 1..M, j in 1..N} X[i,j].ref := val[i,j];

# These identify which SOS constraint each variable belongs to.
let {i in 1..M, j in 1..N} X[i,j].sosno := i;
\end{verbatim}
\end{colorverb}

\subsectionH{From {\tt Gams}}{sec:run_gams}
Thanks to the \footlink{http://projects.coin-or.org/GAMSlinks}{GAMSlinks} project, 
Bonmin is available in {\tt Gams} since release 22.5 of the \footlink{http://www.gams.com/}{\tt GAMS} modeling system. 
The system is available for \footlink{http://download.gams-software.com/}{download from GAMS}. Without buying a license it works as a demo with limited capabilities. Documentation for using \Bonmin\ in {\tt GAMS} is available
\latexhtml{ at
$$
         \mbox{\tt http://www.gams.com/solvers/coin.pdf}
$$
}{\href{http://www.gams.com/solvers/coin.pdf}{here}.}


\subsectionH{From a C/C++ program}{sec:run_cpp}
\Bonmin\ can also be run from within a C/C++ program if the user codes
the functions to compute first- and second-order derivatives.
An example of such a program is available in the subdirectory {\tt CppExample} of
the {\tt examples} directory. For further explanations, please refer to this example and to the reference manual.



\begin{PageSummary}
\PageName{Setting Options}
\PageSection{Passing options to \Bonmin }{sec:opt_opt}
\PageSection{List of options}{sec:options_list}
\PageSection{Getting good solutions to nonconvex problems}{sec:opt_nonconv}
\PageSection{Notes on \Ipopt\ options}{sec:opt_ipopt}
\end{PageSummary}


\PageTitle{Options}{sec:opt}
\subsectionH{Passing options to \Bonmin}{sec:opt_opt}
Options in \Bonmin\ can be set in several different ways.

First, you can set options by putting them in a file called {\tt
bonmin.opt} in the directory where {\tt bonmin} is executing. If you
are familiar with the file
\href{\IpoptDoc{50}}{\tt
ipopt.opt} (formerly named {\tt PARAMS.DAT}) in {\tt Ipopt}, the
syntax of the {\tt bonmin.opt} is similar. For those not familiar
with {\tt ipopt.opt}, the syntax is simply to put the name of the
option followed by its value, with no more than two options on a
single line. Anything on a line after a \# symbol is ignored (i.e.,
treated as a comment).

Note that \Bonmin\ sets options for {\tt
Ipopt}. If you want to set options for {\tt Ipopt} (when used inside \Bonmin) you have to set them
in the file {\tt bonmin.opt} (the standard {\tt Ipopt} option file {\tt ipopt.opt}
is not read by \Bonmin.)
For a list and a description of all the {\tt Ipopt} options, the
reader may refer to the
\footlink{\IpoptDoc{54}}{documentation
of {\tt
Ipopt}}.

Since {\tt bonmin.opt} contains both {\tt Ipopt} and \Bonmin\ options, for clarity
all \Bonmin\ options should be preceded with the prefix ``{\tt bonmin.}'' in {\tt bonmin.opt}~.
Note that some options can also be passed to the MILP subsolver used by \Bonmin\
in the outer approximation decomposition
and the hybrid (see Subsection \ref{sec:milp_opt}).\\

The most important option in \Bonmin\ is the choice of the solution
algorithm. This can be set by using the option named {\tt
bonmin.algorithm} which can be set to {\tt B-BB}, {\tt B-OA}, {\tt
B-QG}, or {\tt B-Hyb} (it's default value is {\tt B-BB}). Depending
on the value of this option, certain other options may be available
or not. \latexhtml{Table \ref{tab:options} gives t}{T}he list of options together
with their types, default values and availability in each of the
four algorithms\latexhtml{}{ can be found \href{\OptListPage \#sec:options_list}{here}}. The column labeled `type' indicates the type of the
parameter (`F' stands for float, `I' for integer, and `S' for
string). The column labeled `default' indicates the global default
value. Then for each of the algorithms {\tt B-BB}, {\tt B-OA},
{\tt B-QG}, {\tt B-Hyb}, {\tt B-Ecp}, and {\tt B-iFP} `$\surd$' indicates that the option is
available for that particular algorithm
while `$-$' indicates that it is not.\\

An example of a {\tt bonmin.opt} file including all the options
with their default values is located in the {\tt Test}
sub-directory.

A small example is as follows:
\begin{verbatim}
   bonmin.bb_log_level 4
   bonmin.algorithm B-BB
   print_level 6
\end{verbatim}
This sets the level of output of the branch-and-bound in \Bonmin\ to $4$, the algorithm to branch-and-bound
and the output level for {\tt Ipopt} to $6$.\\

When \Bonmin\ is run from within {\tt Ampl}, another way to set
an option is through the
internal {\tt Ampl} command {\tt options}.
For example
\begin{verbatim}
options bonmin_options "bonmin.bb_log-level 4 \
                  bonmin.algorithm B-BB print_level 6";
\end{verbatim}
has the same affect as the {\tt bonmin.opt} example above.
Note that any \Bonmin\ option specified in the file {\tt bonmin.opt}
overrides any setting of that option from within {\t Ampl}.\\

A third way is to set options directly in the C/C++ code when
running \Bonmin\ from inside a C/C++ program as is explained in the reference manual.

A detailed description of all of the \Bonmin\ options is given \latexhtml{in Appendix \ref{sec:optList}}{\href{\OptListPage \#sec:options_list}{here}}.
In the following, we give some more details on options for the MILP subsolver and
on the options specifically designed
for nonconvex problems.\\

\latexhtml{
\topcaption{\label{tab:options} 
List of options and compatibility with the different algorithms.
}
\tablehead{\hline 
Option & type & {\tt B-BB} & {\tt B-OA} & {\tt B-QG} & {\tt B-Hyb} & {\tt B-Ecp} & {\tt B-iFP} & {\tt Cbc\_Par} \\
\hline
\hline}
\tabletail{\hline \multicolumn{9}{|c|}{continued on next page}\\\hline}
\tablelasttail{\hline}
{\tiny
\begin{xtabular}{|l|r|r|r|r|r|r|r|r|}
\hline
\multicolumn{1}{|c}{} & \multicolumn{8}{l|}{Algorithm choice}\\
\hline
algorithm& S& $\surd$& $\surd$& $\surd$& $\surd$& $\surd$& $\surd$& $\surd$\\
\hline
\multicolumn{1}{|c}{} & \multicolumn{8}{l|}{Branch-and-bound options}\\
\hline
allowable\_fraction\_gap& F& $\surd$& $\surd$& $\surd$& $\surd$& $\surd$& $\surd$& $\surd$\\
allowable\_gap& F& $\surd$& $\surd$& $\surd$& $\surd$& $\surd$& $\surd$& $\surd$\\
cutoff& F& $\surd$& $\surd$& $\surd$& $\surd$& $\surd$& $\surd$& $\surd$\\
cutoff\_decr& F& $\surd$& $\surd$& $\surd$& $\surd$& $\surd$& $\surd$& $\surd$\\
enable\_dynamic\_nlp& S& $\surd$& -& -& -& -& -& -\\
integer\_tolerance& F& $\surd$& $\surd$& $\surd$& $\surd$& $\surd$& $\surd$& $\surd$\\
iteration\_limit& I& $\surd$& $\surd$& $\surd$& $\surd$& $\surd$& $\surd$& $\surd$\\
nlp\_failure\_behavior& S& $\surd$& -& -& -& -& -& -\\
node\_comparison& S& $\surd$& $\surd$& $\surd$& $\surd$& $\surd$& $\surd$& -\\
node\_limit& I& $\surd$& $\surd$& $\surd$& $\surd$& $\surd$& $\surd$& $\surd$\\
num\_cut\_passes& I& -& -& $\surd$& $\surd$& $\surd$& -& -\\
num\_cut\_passes\_at\_root& I& -& -& $\surd$& $\surd$& $\surd$& -& -\\
number\_before\_trust& I& $\surd$& $\surd$& $\surd$& $\surd$& $\surd$& $\surd$& $\surd$\\
number\_strong\_branch& I& $\surd$& $\surd$& $\surd$& $\surd$& $\surd$& $\surd$& $\surd$\\
random\_generator\_seed& I& $\surd$& $\surd$& $\surd$& $\surd$& $\surd$& $\surd$& $\surd$\\
read\_solution\_file& S& $\surd$& $\surd$& $\surd$& $\surd$& $\surd$& $\surd$& $\surd$\\
solution\_limit& I& $\surd$& $\surd$& $\surd$& $\surd$& $\surd$& $\surd$& $\surd$\\
sos\_constraints& S& $\surd$& $\surd$& $\surd$& $\surd$& $\surd$& $\surd$& -\\
time\_limit& F& $\surd$& $\surd$& $\surd$& $\surd$& $\surd$& $\surd$& $\surd$\\
tree\_search\_strategy& S& $\surd$& $\surd$& $\surd$& $\surd$& $\surd$& $\surd$& -\\
variable\_selection& S& $\surd$& -& $\surd$& $\surd$& $\surd$& -& -\\
\hline
\multicolumn{1}{|c}{} & \multicolumn{8}{l|}{MILP cutting planes in hybrid}\\
\hline
2mir\_cuts& I& -& $\surd$& $\surd$& $\surd$& $\surd$& $\surd$& $\surd$\\
Gomory\_cuts& I& -& $\surd$& $\surd$& $\surd$& $\surd$& $\surd$& $\surd$\\
clique\_cuts& I& -& $\surd$& $\surd$& $\surd$& $\surd$& $\surd$& $\surd$\\
cover\_cuts& I& -& $\surd$& $\surd$& $\surd$& $\surd$& $\surd$& $\surd$\\
flow\_cover\_cuts& I& -& $\surd$& $\surd$& $\surd$& $\surd$& $\surd$& $\surd$\\
lift\_and\_project\_cuts& I& -& $\surd$& $\surd$& $\surd$& $\surd$& $\surd$& $\surd$\\
mir\_cuts& I& -& $\surd$& $\surd$& $\surd$& $\surd$& $\surd$& $\surd$\\
reduce\_and\_split\_cuts& I& -& $\surd$& $\surd$& $\surd$& $\surd$& $\surd$& $\surd$\\
\hline
\multicolumn{1}{|c}{} & \multicolumn{8}{l|}{MINLP Heuristics}\\
\hline
feasibility\_pump\_objective\_norm& I& $\surd$& $\surd$& $\surd$& $\surd$& $\surd$& $\surd$& -\\
heuristic\_RINS& S& $\surd$& $\surd$& $\surd$& $\surd$& $\surd$& $\surd$& -\\
heuristic\_dive\_MIP\_fractional& S& $\surd$& $\surd$& $\surd$& $\surd$& $\surd$& $\surd$& -\\
heuristic\_dive\_MIP\_vectorLength& S& $\surd$& $\surd$& $\surd$& $\surd$& $\surd$& $\surd$& -\\
heuristic\_dive\_fractional& S& $\surd$& $\surd$& $\surd$& $\surd$& $\surd$& $\surd$& -\\
heuristic\_dive\_vectorLength& S& $\surd$& $\surd$& $\surd$& $\surd$& $\surd$& $\surd$& -\\
heuristic\_feasibility\_pump& S& $\surd$& $\surd$& $\surd$& $\surd$& $\surd$& $\surd$& -\\
pump\_for\_minlp& S& $\surd$& $\surd$& $\surd$& $\surd$& $\surd$& $\surd$& -\\
\hline
\multicolumn{1}{|c}{} & \multicolumn{8}{l|}{Nlp solution robustness}\\
\hline
max\_consecutive\_failures& I& $\surd$& -& -& -& -& -& -\\
max\_random\_point\_radius& F& $\surd$& -& -& -& -& -& -\\
num\_iterations\_suspect& I& $\surd$& $\surd$& $\surd$& $\surd$& $\surd$& $\surd$& $\surd$\\
num\_retry\_unsolved\_random\_point& I& $\surd$& $\surd$& $\surd$& $\surd$& $\surd$& $\surd$& $\surd$\\
random\_point\_perturbation\_interval& F& $\surd$& -& -& -& -& -& -\\
random\_point\_type& S& $\surd$& -& -& -& -& -& -\\
resolve\_on\_small\_infeasibility& F& $\surd$& $\surd$& $\surd$& $\surd$& $\surd$& $\surd$& $\surd$\\
\hline
\multicolumn{1}{|c}{} & \multicolumn{8}{l|}{Nlp solve options in B-Hyb}\\
\hline
nlp\_solve\_frequency& I& -& -& -& $\surd$& -& -& -\\
nlp\_solve\_max\_depth& I& -& -& -& $\surd$& -& -& -\\
nlp\_solves\_per\_depth& F& -& -& -& $\surd$& -& -& -\\
\hline
\multicolumn{1}{|c}{} & \multicolumn{8}{l|}{Options for MILP solver}\\
\hline
cpx\_parallel\_strategy& I& -& -& -& -& -& -& $\surd$\\
milp\_log\_level& I& -& -& -& -& -& -& $\surd$\\
milp\_solver& S& -& -& -& -& -& -& $\surd$\\
milp\_strategy& S& -& -& -& -& -& -& $\surd$\\
number\_cpx\_threads& I& -& -& -& -& -& -& $\surd$\\
\hline
\multicolumn{1}{|c}{} & \multicolumn{8}{l|}{Options for OA decomposition}\\
\hline
oa\_decomposition& S& -& -& $\surd$& $\surd$& $\surd$& -& -\\
oa\_log\_frequency& F& $\surd$& -& -& $\surd$& $\surd$& -& -\\
oa\_log\_level& I& $\surd$& -& -& $\surd$& $\surd$& -& -\\
\hline
\multicolumn{1}{|c}{} & \multicolumn{8}{l|}{Options for ecp cuts generation}\\
\hline
ecp\_abs\_tol& F& -& -& $\surd$& $\surd$& -& -& -\\
ecp\_max\_rounds& I& -& -& $\surd$& $\surd$& -& -& -\\
ecp\_probability\_factor& F& -& -& $\surd$& $\surd$& -& -& -\\
ecp\_rel\_tol& F& -& -& $\surd$& $\surd$& -& -& -\\
filmint\_ecp\_cuts& I& -& -& $\surd$& $\surd$& -& -& -\\
\hline
\multicolumn{1}{|c}{} & \multicolumn{8}{l|}{Options for feasibility checker using OA cuts}\\
\hline
feas\_check\_cut\_types& S& -& -& $\surd$& $\surd$& $\surd$& -& -\\
feas\_check\_discard\_policy& S& -& -& $\surd$& $\surd$& $\surd$& -& -\\
generate\_benders\_after\_so\_many\_oa& I& -& -& $\surd$& $\surd$& $\surd$& -& -\\
\hline
\multicolumn{1}{|c}{} & \multicolumn{8}{l|}{Options for feasibility pump}\\
\hline
fp\_log\_frequency& F& -& -& $\surd$& $\surd$& -& -& -\\
fp\_log\_level& I& -& -& $\surd$& $\surd$& -& -& -\\
fp\_pass\_infeasible& S& $\surd$& $\surd$& $\surd$& $\surd$& $\surd$& $\surd$& $\surd$\\
\hline
\multicolumn{1}{|c}{} & \multicolumn{8}{l|}{Options for non-convex problems}\\
\hline
coeff\_var\_threshold& F& $\surd$& -& -& -& -& -& -\\
dynamic\_def\_cutoff\_decr& S& $\surd$& -& -& -& -& -& -\\
first\_perc\_for\_cutoff\_decr& F& $\surd$& -& -& -& -& -& -\\
max\_consecutive\_infeasible& I& $\surd$& -& -& -& -& -& -\\
num\_resolve\_at\_infeasibles& I& $\surd$& -& -& -& -& -& -\\
num\_resolve\_at\_node& I& $\surd$& -& -& -& -& -& -\\
num\_resolve\_at\_root& I& $\surd$& -& -& -& -& -& -\\
second\_perc\_for\_cutoff\_decr& F& $\surd$& -& -& -& -& -& -\\
\hline
\multicolumn{1}{|c}{} & \multicolumn{8}{l|}{Outer Approximation cuts generation}\\
\hline
add\_only\_violated\_oa& S& -& $\surd$& $\surd$& $\surd$& $\surd$& $\surd$& $\surd$\\
oa\_cuts\_log\_level& I& -& $\surd$& $\surd$& $\surd$& $\surd$& $\surd$& $\surd$\\
oa\_cuts\_scope& S& -& $\surd$& $\surd$& $\surd$& $\surd$& $\surd$& $\surd$\\
oa\_rhs\_relax& F& -& $\surd$& $\surd$& $\surd$& $\surd$& $\surd$& $\surd$\\
tiny\_element& F& -& $\surd$& $\surd$& $\surd$& $\surd$& $\surd$& $\surd$\\
very\_tiny\_element& F& -& $\surd$& $\surd$& $\surd$& $\surd$& $\surd$& $\surd$\\
\hline
\multicolumn{1}{|c}{} & \multicolumn{8}{l|}{Output and log-level options}\\
\hline
bb\_log\_interval& I& $\surd$& $\surd$& $\surd$& $\surd$& $\surd$& $\surd$& $\surd$\\
bb\_log\_level& I& $\surd$& $\surd$& $\surd$& $\surd$& $\surd$& $\surd$& $\surd$\\
lp\_log\_level& I& -& $\surd$& $\surd$& $\surd$& $\surd$& $\surd$& $\surd$\\
nlp\_log\_at\_root& I& $\surd$& $\surd$& $\surd$& $\surd$& $\surd$& $\surd$& -\\
\hline
\multicolumn{1}{|c}{} & \multicolumn{8}{l|}{Strong branching setup}\\
\hline
candidate\_sort\_criterion& S& $\surd$& $\surd$& $\surd$& $\surd$& $\surd$& $\surd$& -\\
maxmin\_crit\_have\_sol& F& $\surd$& $\surd$& $\surd$& $\surd$& $\surd$& $\surd$& -\\
maxmin\_crit\_no\_sol& F& $\surd$& $\surd$& $\surd$& $\surd$& $\surd$& $\surd$& -\\
min\_number\_strong\_branch& I& $\surd$& $\surd$& $\surd$& $\surd$& $\surd$& $\surd$& -\\
number\_before\_trust\_list& I& $\surd$& $\surd$& $\surd$& $\surd$& $\surd$& $\surd$& -\\
number\_look\_ahead& I& $\surd$& $\surd$& $\surd$& $\surd$& $\surd$& -& -\\
number\_strong\_branch\_root& I& $\surd$& $\surd$& $\surd$& $\surd$& $\surd$& $\surd$& -\\
setup\_pseudo\_frac& F& $\surd$& $\surd$& $\surd$& $\surd$& $\surd$& $\surd$& -\\
trust\_strong\_branching\_for\_pseudo\_cost& S& $\surd$& $\surd$& $\surd$& $\surd$& $\surd$& $\surd$& -\\
\hline
\multicolumn{1}{|c}{} & \multicolumn{8}{l|}{nlp interface option}\\
\hline
file\_solution& S& $\surd$& $\surd$& $\surd$& $\surd$& $\surd$& $\surd$& $\surd$\\
nlp\_log\_level& I& $\surd$& $\surd$& $\surd$& $\surd$& $\surd$& $\surd$& $\surd$\\
nlp\_solver& S& $\surd$& $\surd$& $\surd$& $\surd$& $\surd$& $\surd$& $\surd$\\
warm\_start& S& $\surd$& -& -& -& -& -& -\\
\hline
\end{xtabular}
}

}{
}

\subsectionH{Passing options to local search based heuristics and oa generators}{sec:sub_solvers}
\label{sec:sub_solvers}
Several parts of the algorithms in \Bonmin\ are based on solving a simplified version of the problem with another instance of \Bonmin:
Outer Approximation Decomposition (called in {\tt B-Hyb} at the root node)
and Feasibility Pump for MINLP (called in B-Hyb or B-BB at the root node), RINS, RENS, Local Branching.

In all these cases, one can pass options to the sub-algorithm used through the bonmin.opt file. The basic principle is
that the ``bonmin.'' prefix  is replaced with a prefix that identifies the sub-algorithm used:
\begin{itemize}
\item to pass options to Outer Approximation Decomposition: {\tt oa\_decomposition.},
\item to pass options to Feasibility Pump for MINLP: {\tt pump\_for\_minlp.},
\item to pass options to RINS: {\tt rins.},
\item to pass options to RENS: {\tt rens.},
\item to pass options to Local Branching: {\tt local\_branch}.
\end{itemize}


For example, we may want to run a maximum of 60 seconds of FP for MINLP until 6 solutions are found at the beginning of the hybrid algorithm. To do so 
we set the following option in {\tt bonmin.opt}
\begin{verbatim}
bonmin.algorithm B-Hyb

bonmin.pump_for_minlp yes      #Tells to run fp for MINLP
pump_for_minlp.time_limit 60 #set a time limit for the pump
pump_for_minlp.solution_limit 6 # set a solution limit

\end{verbatim}

Note that the actual solution and time limit will be the minimum of the global limits set for \Bonmin.

A slightly more complicated set of options may be used when using RINS. Say for example that we want to run RINS inside B-BB. Each time RINS is called we want
to solve the small-size MINLP generated using B-QG (we may run any algorithm available in \Bonmin for solving an MINLP) and want to stop as soon as B-QG found 1 solution.
We set the following options in bonmin.opt

\begin{verbatim}
bonmin.algorithm B-BB

bonmin.rins yes
rins.algorithm B-QG
rins.solution_limit 1

\end{verbatim}
This example shows that it is possible to set any option used in the sub-algorithm to be different than the one used for the main algorithm.


In the context of OA and FP for MINLP, a standard MILP solver is used.
Several option are available for configuring this MILP solver.
\Bonmin\ allows a choice of different MILP solvers through the option
{\tt bonmin.milp\_subsolver}. Values for this option are: {\tt Cbc\_D} which uses {\tt Cbc} with its
default settings, {\tt Cplex} which uses {\tt Cplex} with its default settings, and
{\tt Cbc\_Par} which uses a version of {\tt Cbc} that can be parameterized by the user.
The options that can be set in {\tt Cbc\_Par} are the number of strong-branching candidates,
the number of branches before pseudo costs are to be trusted, and the frequency of the various cut generators
(these options are signaled in Table \ref{tab:options}).

\subsectionH{Getting good solutions to nonconvex problems}{sec:opt_nonconv}
\label{sec:non_convex}
To solve a problem with non-convex constraints, one should only use the branch-and-bound algorithm {\tt B-BB}.


A few options have been designed in \Bonmin\ specifically to treat
problems that do not have a convex continuous relaxation.
In such problems, the solutions obtained from {\tt Ipopt} are
not necessarily globally optimal, but are only locally optimal. Also the outer-approximation
constraints are not necessarily valid inequalities for the problem.

No specific heuristic method for treating nonconvex problems is implemented
yet within the OA framework.
But for the pure branch-and-bound {\tt B-BB}, we implemented a few options having
in mind that lower bounds provided by {\tt Ipopt} should not be trusted, and with the goal of
trying to get good solutions. Such options are at a very experimental stage.

First, in the context of nonconvex problems, {\tt Ipopt} may find different local optima when started
from different starting points. The two options {\tt num\_re\-solve\_at\_root} and {\tt num\_resolve\_at\_node}
allow for solving the root node or each node of the tree, respectively, with a user-specified
number of different randomly-chosen
starting points, saving the best solution found. Note that the function to generate a random starting point
is very na\"{\i}ve: it chooses a random point (uniformly) between the bounds provided for the variable.
In particular if there are some functions
that can not be evaluated at some points of the domain, it may pick such points,
 and so it is not robust in that respect.

Secondly, since the solution given by {\tt Ipopt} does not truly give a lower bound, we allow for
changing the fathoming rule
to continue branching even if the solution value to the current node is worse
than the best-known
solution. This is achieved by setting {\tt allowable\_gap}
and {\tt allowable\_fraction\_gap} and {\tt cutoff\_decr} to negative values.

\subsectionH{Notes on {\tt Ipopt} options}{sec:opt_ipopt}
\label{sec:opt_ipopt}
\Ipopt\ has a very large number of options, to get a complete description of them, you
should refer to the \Ipopt\ manual.
Here we only mention and explain some of the options that have been more important to us, so far,
in developing and using \Bonmin.
\subsubsection{Default options changed by \Bonmin}
\Ipopt\ has been tailored to be more efficient when used in the context of the
solution of a MINLP problem. In particular, we have tried to
improve \Ipopt's warm-starting capabilities and its ability to prove quickly that a subproblem
is infeasible. For ordinary NLP problems, \Ipopt\ does not use these options
by default, but \Bonmin\ automatically changes these options from their default values.

Note that options set by the user in {\tt bonmin.opt} will override these
settings.

\paragraph{{\tt mu\_strategy} and {\tt mu\_oracle}} are set, respectively, to
{\tt adaptive} and {\tt probing} by default (these are newly implemented strategies in \Ipopt\
for updating the barrier parameter \mycite{NocedalAdaptive}{Nocedal2004} which we have found to be
more efficient in the context of MINLP).
\paragraph{{\tt gamma\_phi} and {\tt gamma\_theta}} are set to $10^{-8}$ and $10^{-4}$
respectively. This has the effect of reducing the size of the filter in the line search performed by \Ipopt.

\paragraph{\tt required\_infeasibility\_reduction} is set to $0.1$.
This increases the required infeasibility reduction when \Ipopt\ enters the
restoration phase and should thus help to
detect infeasible problems faster.

\paragraph{\tt expect\_infeasible\_problem} is set to {\tt yes}, which enables some heuristics
to detect infeasible problems faster.

\paragraph{\tt warm\_start\_init\_point} is set to {\tt yes} when a full primal/dual starting
point is available (generally all the optimizations after the continuous relaxation has been solved).

\paragraph{\tt print\_level} is set to $0$ by default to turn off \Ipopt\ output.
\subsubsection{Some useful \Ipopt\ options}
\paragraph{bound\_relax\_factor} is by default set to $10^{-8}$ in \Ipopt. All of the bounds
of the problem are relaxed by this factor. This may cause some trouble
when constraint functions can only be evaluated within their bounds.
In such cases, this
option should be set to 0.

%\section{{\Bonmin} output}
%\Bonmin uses a number of optimization tools ({\
%tt Ipopt}, {\tt Cbc}, {\tt Clp}, \ldots) which all have there own output.
%As a consequence the output is very diverse and may be a little difficult to read.\\
%
%The output level for each of the building blocks of \Bonmin can be set through options (see Appendix \ref{app:opt_loglevel})
%Here we briefly describes the output given at the different log level for each of those building blocks.
%\subsection{BB\_log_level}
%This corresponds to the output from the main branch-and-bound process in B-BB, B-QG, B-Hyb
%(this is not available in the B-OA).
%The output here comes from {\tt Cbc}, and each line of output has the prefix {\tt Cbc} and is followed by a number indicating the message type.
%
%\subsection{
\begin{PageSummary}
\end{PageSummary}
\begin{thebibliography}{10}


\bibitem{Betal}
P.~Bonami, A.~W\"achter, L.T.~Biegler, A.R.~Conn, G.~Cornu\'ejols,
I.E.~Grossmann, C.D.~Laird, J.~Lee, A.~Lodi, F.~Margot and
N.~Sawaya.
\newblock An algorithmic framework for convex mixed integer nonlinear programs. IBM Research Report RC23771, Oct. 2005.

\bibitem{Gupta80Nonlinear}
O.K.~Gupta and V.~Ravindran.
\newblock Branch and bound experiments in convex nonlinear integer programming.
\newblock {\em Management Science}, 31:1533--1546, 1985.

%\bibitem{fiacco} A.V.~Fiacco.
%\newblock Introduction to Sensitivity and
%    Stability Analysis in Nonlinear Programming.
%\newblock {\em Academic Press, New
%  York} (1983)

\bibitem{DG}
M.~Duran and I.E.~Grossmann.
\newblock An outer-approximation algorithm for a class of mixed-integer nonlinear programs.
\newblock {\em Mathematical Programming}, 36:307--339, 1986.

\bibitem{FL}
R.~Fletcher and S.~Leyffer.
\newblock Solving mixed integer nonlinear programs by outer approximation.
\newblock {\em Mathematical Programming}, 66:327--349, 1994.

\bibitem{Gay}
D.M.~Gay.
\newblock Writing \texttt{\bf .nl} files.
\newblock Sandia National Laboratories, Technical Report No. 2005-7907P, 2005.

\bibitem{QG}
I.~Quesada and I.E.~Grossmann.
\newblock An {LP/NLP} based branched and bound algorithm for convex {MINLP} optimization problems.
\newblock {\em Computers and Chemical Engineering}, 16:937--947, 1992.

\bibitem{AMPL}
R.~Fourer and D.M.~Gay and B.W.~Kernighan.
\newblock AMPL: A Modeling Language for Mathematical
Programming, Second Edition,
\newblock Duxbury Press Brooks Cole Publishing Co., 2003.


\bibitem{NocedalAdaptive}
J.~Nocedal, A.~W\"achter, and R.~A. Waltz.
\newblock Adaptive Barrier Strategies for Nonlinear Interior Methods.
\newblock Research Report RC 23563, IBM T. J. Watson Research Center, Yorktown, USA (March 2005; revised January 2006)

\bibitem{AndreasIpopt}
A.~W\"achter and L.~T.~Biegler.
\newblock On the Implementation of a Primal-Dual Interior Point Filter Line Search Algorithm for Large-Scale Nonlinear Programming.
\newblock Mathematical Programming 106(1), pp. 25-57, 2006
\end{thebibliography}

\appendix
\latexhtml{\section{List of \Bonmin\ options}
\label{sec:optList}

%%% Local Variables:
%%% mode: latex
%%% TeX-master: "BONMIN_UsersManual"
%%% End:


\begin{PageSummary}
\PageName{List of \Bonmin\ options}
\end{PageSummary}

\latexhtml{}{
\HCode{
<table border="1">
<tr>
<td>Option </td>
<td> type </td>
<td> B-BB</td>
<td> B-OA</td>
<td> B-QG</td>
<td> B-Hyb</td>
</tr>
<tr>   <th colspan=9> <a href="#sec:Algorithm_choice">Algorithm choice</a> </th>
</tr>
<tr>
<td> <a href="#sec:algorithm">algorithm</a> </td>
<td>S</td>
<td> +</td>
<td>+</td>
<td>+</td>
<td>+</td>
</tr>
<tr>   <th colspan=9> <a href="#sec:Branch-and-bound_options">Branch-and-bound options</a> </th>
</tr>
<tr>
<td> <a href="#sec:allowable_fraction_gap">allowable_fraction_gap</a> </td>
<td>F</td>
<td> +</td>
<td>+</td>
<td>+</td>
<td>+</td>
</tr>
<tr>
<td> <a href="#sec:allowable_gap">allowable_gap</a> </td>
<td>F</td>
<td> +</td>
<td>+</td>
<td>+</td>
<td>+</td>
</tr>
<tr>
<td> <a href="#sec:cutoff">cutoff</a> </td>
<td>F</td>
<td> +</td>
<td>+</td>
<td>+</td>
<td>+</td>
</tr>
<tr>
<td> <a href="#sec:cutoff_decr">cutoff_decr</a> </td>
<td>F</td>
<td> +</td>
<td>+</td>
<td>+</td>
<td>+</td>
</tr>
<tr>
<td> <a href="#sec:enable_dynamic_nlp">enable_dynamic_nlp</a> </td>
<td>S</td>
<td> +</td>
<td>-</td>
<td>-</td>
<td>-</td>
</tr>
<tr>
<td> <a href="#sec:integer_tolerance">integer_tolerance</a> </td>
<td>F</td>
<td> +</td>
<td>+</td>
<td>+</td>
<td>+</td>
</tr>
<tr>
<td> <a href="#sec:iteration_limit">iteration_limit</a> </td>
<td>I</td>
<td> +</td>
<td>+</td>
<td>+</td>
<td>+</td>
</tr>
<tr>
<td> <a href="#sec:nlp_failure_behavior">nlp_failure_behavior</a> </td>
<td>S</td>
<td> +</td>
<td>-</td>
<td>-</td>
<td>-</td>
</tr>
<tr>
<td> <a href="#sec:node_comparison">node_comparison</a> </td>
<td>S</td>
<td> +</td>
<td>+</td>
<td>+</td>
<td>+</td>
</tr>
<tr>
<td> <a href="#sec:node_limit">node_limit</a> </td>
<td>I</td>
<td> +</td>
<td>+</td>
<td>+</td>
<td>+</td>
</tr>
<tr>
<td> <a href="#sec:num_cut_passes">num_cut_passes</a> </td>
<td>I</td>
<td> -</td>
<td>-</td>
<td>+</td>
<td>+</td>
</tr>
<tr>
<td> <a href="#sec:num_cut_passes_at_root">num_cut_passes_at_root</a> </td>
<td>I</td>
<td> -</td>
<td>-</td>
<td>+</td>
<td>+</td>
</tr>
<tr>
<td> <a href="#sec:number_before_trust">number_before_trust</a> </td>
<td>I</td>
<td> +</td>
<td>+</td>
<td>+</td>
<td>+</td>
</tr>
<tr>
<td> <a href="#sec:number_strong_branch">number_strong_branch</a> </td>
<td>I</td>
<td> +</td>
<td>+</td>
<td>+</td>
<td>+</td>
</tr>
<tr>
<td> <a href="#sec:random_generator_seed">random_generator_seed</a> </td>
<td>I</td>
<td> +</td>
<td>+</td>
<td>+</td>
<td>+</td>
</tr>
<tr>
<td> <a href="#sec:read_solution_file">read_solution_file</a> </td>
<td>S</td>
<td> +</td>
<td>+</td>
<td>+</td>
<td>+</td>
</tr>
<tr>
<td> <a href="#sec:solution_limit">solution_limit</a> </td>
<td>I</td>
<td> +</td>
<td>+</td>
<td>+</td>
<td>+</td>
</tr>
<tr>
<td> <a href="#sec:sos_constraints">sos_constraints</a> </td>
<td>S</td>
<td> +</td>
<td>+</td>
<td>+</td>
<td>+</td>
</tr>
<tr>
<td> <a href="#sec:time_limit">time_limit</a> </td>
<td>F</td>
<td> +</td>
<td>+</td>
<td>+</td>
<td>+</td>
</tr>
<tr>
<td> <a href="#sec:tree_search_strategy">tree_search_strategy</a> </td>
<td>S</td>
<td> +</td>
<td>+</td>
<td>+</td>
<td>+</td>
</tr>
<tr>
<td> <a href="#sec:variable_selection">variable_selection</a> </td>
<td>S</td>
<td> +</td>
<td>-</td>
<td>+</td>
<td>+</td>
</tr>
<tr>   <th colspan=9> <a href="#sec:MILP_cutting_planes_in_hybrid">MILP cutting planes in hybrid</a> </th>
</tr>
<tr>
<td> <a href="#sec:2mir_cuts">2mir_cuts</a> </td>
<td>I</td>
<td> -</td>
<td>+</td>
<td>+</td>
<td>+</td>
</tr>
<tr>
<td> <a href="#sec:Gomory_cuts">Gomory_cuts</a> </td>
<td>I</td>
<td> -</td>
<td>+</td>
<td>+</td>
<td>+</td>
</tr>
<tr>
<td> <a href="#sec:clique_cuts">clique_cuts</a> </td>
<td>I</td>
<td> -</td>
<td>+</td>
<td>+</td>
<td>+</td>
</tr>
<tr>
<td> <a href="#sec:cover_cuts">cover_cuts</a> </td>
<td>I</td>
<td> -</td>
<td>+</td>
<td>+</td>
<td>+</td>
</tr>
<tr>
<td> <a href="#sec:flow_cover_cuts">flow_cover_cuts</a> </td>
<td>I</td>
<td> -</td>
<td>+</td>
<td>+</td>
<td>+</td>
</tr>
<tr>
<td> <a href="#sec:lift_and_project_cuts">lift_and_project_cuts</a> </td>
<td>I</td>
<td> -</td>
<td>+</td>
<td>+</td>
<td>+</td>
</tr>
<tr>
<td> <a href="#sec:mir_cuts">mir_cuts</a> </td>
<td>I</td>
<td> -</td>
<td>+</td>
<td>+</td>
<td>+</td>
</tr>
<tr>
<td> <a href="#sec:reduce_and_split_cuts">reduce_and_split_cuts</a> </td>
<td>I</td>
<td> -</td>
<td>+</td>
<td>+</td>
<td>+</td>
</tr>
<tr>   <th colspan=9> <a href="#sec:MINLP_Heuristics">MINLP Heuristics</a> </th>
</tr>
<tr>
<td> <a href="#sec:feasibility_pump_objective_norm">feasibility_pump_objective_norm</a> </td>
<td>I</td>
<td> +</td>
<td>+</td>
<td>+</td>
<td>+</td>
</tr>
<tr>
<td> <a href="#sec:heuristic_RINS">heuristic_RINS</a> </td>
<td>S</td>
<td> +</td>
<td>+</td>
<td>+</td>
<td>+</td>
</tr>
<tr>
<td> <a href="#sec:heuristic_dive_MIP_fractional">heuristic_dive_MIP_fractional</a> </td>
<td>S</td>
<td> +</td>
<td>+</td>
<td>+</td>
<td>+</td>
</tr>
<tr>
<td> <a href="#sec:heuristic_dive_MIP_vectorLength">heuristic_dive_MIP_vectorLength</a> </td>
<td>S</td>
<td> +</td>
<td>+</td>
<td>+</td>
<td>+</td>
</tr>
<tr>
<td> <a href="#sec:heuristic_dive_fractional">heuristic_dive_fractional</a> </td>
<td>S</td>
<td> +</td>
<td>+</td>
<td>+</td>
<td>+</td>
</tr>
<tr>
<td> <a href="#sec:heuristic_dive_vectorLength">heuristic_dive_vectorLength</a> </td>
<td>S</td>
<td> +</td>
<td>+</td>
<td>+</td>
<td>+</td>
</tr>
<tr>
<td> <a href="#sec:heuristic_feasibility_pump">heuristic_feasibility_pump</a> </td>
<td>S</td>
<td> +</td>
<td>+</td>
<td>+</td>
<td>+</td>
</tr>
<tr>
<td> <a href="#sec:pump_for_minlp">pump_for_minlp</a> </td>
<td>S</td>
<td> +</td>
<td>+</td>
<td>+</td>
<td>+</td>
</tr>
<tr>   <th colspan=9> <a href="#sec:Nlp_solution_robustness">Nlp solution robustness</a> </th>
</tr>
<tr>
<td> <a href="#sec:max_consecutive_failures">max_consecutive_failures</a> </td>
<td>I</td>
<td> +</td>
<td>-</td>
<td>-</td>
<td>-</td>
</tr>
<tr>
<td> <a href="#sec:max_random_point_radius">max_random_point_radius</a> </td>
<td>F</td>
<td> +</td>
<td>-</td>
<td>-</td>
<td>-</td>
</tr>
<tr>
<td> <a href="#sec:num_iterations_suspect">num_iterations_suspect</a> </td>
<td>I</td>
<td> +</td>
<td>+</td>
<td>+</td>
<td>+</td>
</tr>
<tr>
<td> <a href="#sec:num_retry_unsolved_random_point">num_retry_unsolved_random_point</a> </td>
<td>I</td>
<td> +</td>
<td>+</td>
<td>+</td>
<td>+</td>
</tr>
<tr>
<td> <a href="#sec:random_point_perturbation_interval">random_point_perturbation_interval</a> </td>
<td>F</td>
<td> +</td>
<td>-</td>
<td>-</td>
<td>-</td>
</tr>
<tr>
<td> <a href="#sec:random_point_type">random_point_type</a> </td>
<td>S</td>
<td> +</td>
<td>-</td>
<td>-</td>
<td>-</td>
</tr>
<tr>
<td> <a href="#sec:resolve_on_small_infeasibility">resolve_on_small_infeasibility</a> </td>
<td>F</td>
<td> +</td>
<td>+</td>
<td>+</td>
<td>+</td>
</tr>
<tr>   <th colspan=9> <a href="#sec:Nlp_solve_options_in_B-Hyb">Nlp solve options in B-Hyb</a> </th>
</tr>
<tr>
<td> <a href="#sec:nlp_solve_frequency">nlp_solve_frequency</a> </td>
<td>I</td>
<td> -</td>
<td>-</td>
<td>-</td>
<td>+</td>
</tr>
<tr>
<td> <a href="#sec:nlp_solve_max_depth">nlp_solve_max_depth</a> </td>
<td>I</td>
<td> -</td>
<td>-</td>
<td>-</td>
<td>+</td>
</tr>
<tr>
<td> <a href="#sec:nlp_solves_per_depth">nlp_solves_per_depth</a> </td>
<td>F</td>
<td> -</td>
<td>-</td>
<td>-</td>
<td>+</td>
</tr>
<tr>   <th colspan=9> <a href="#sec:Options_for_MILP_solver">Options for MILP solver</a> </th>
</tr>
<tr>
<td> <a href="#sec:cpx_parallel_strategy">cpx_parallel_strategy</a> </td>
<td>I</td>
<td> -</td>
<td>-</td>
<td>-</td>
<td>-</td>
</tr>
<tr>
<td> <a href="#sec:milp_log_level">milp_log_level</a> </td>
<td>I</td>
<td> -</td>
<td>-</td>
<td>-</td>
<td>-</td>
</tr>
<tr>
<td> <a href="#sec:milp_solver">milp_solver</a> </td>
<td>S</td>
<td> -</td>
<td>-</td>
<td>-</td>
<td>-</td>
</tr>
<tr>
<td> <a href="#sec:milp_strategy">milp_strategy</a> </td>
<td>S</td>
<td> -</td>
<td>-</td>
<td>-</td>
<td>-</td>
</tr>
<tr>
<td> <a href="#sec:number_cpx_threads">number_cpx_threads</a> </td>
<td>I</td>
<td> -</td>
<td>-</td>
<td>-</td>
<td>-</td>
</tr>
<tr>   <th colspan=9> <a href="#sec:Options_for_OA_decomposition">Options for OA decomposition</a> </th>
</tr>
<tr>
<td> <a href="#sec:oa_decomposition">oa_decomposition</a> </td>
<td>S</td>
<td> -</td>
<td>-</td>
<td>+</td>
<td>+</td>
</tr>
<tr>
<td> <a href="#sec:oa_log_frequency">oa_log_frequency</a> </td>
<td>F</td>
<td> +</td>
<td>-</td>
<td>-</td>
<td>+</td>
</tr>
<tr>
<td> <a href="#sec:oa_log_level">oa_log_level</a> </td>
<td>I</td>
<td> +</td>
<td>-</td>
<td>-</td>
<td>+</td>
</tr>
<tr>   <th colspan=9> <a href="#sec:Options_for_ecp_cuts_generation">Options for ecp cuts generation</a> </th>
</tr>
<tr>
<td> <a href="#sec:ecp_abs_tol">ecp_abs_tol</a> </td>
<td>F</td>
<td> -</td>
<td>-</td>
<td>+</td>
<td>+</td>
</tr>
<tr>
<td> <a href="#sec:ecp_max_rounds">ecp_max_rounds</a> </td>
<td>I</td>
<td> -</td>
<td>-</td>
<td>+</td>
<td>+</td>
</tr>
<tr>
<td> <a href="#sec:ecp_probability_factor">ecp_probability_factor</a> </td>
<td>F</td>
<td> -</td>
<td>-</td>
<td>+</td>
<td>+</td>
</tr>
<tr>
<td> <a href="#sec:ecp_rel_tol">ecp_rel_tol</a> </td>
<td>F</td>
<td> -</td>
<td>-</td>
<td>+</td>
<td>+</td>
</tr>
<tr>
<td> <a href="#sec:filmint_ecp_cuts">filmint_ecp_cuts</a> </td>
<td>I</td>
<td> -</td>
<td>-</td>
<td>+</td>
<td>+</td>
</tr>
<tr>   <th colspan=9> <a href="#sec:Options_for_feasibility_checker_using_OA_cuts">Options for feasibility checker using OA cuts</a> </th>
</tr>
<tr>
<td> <a href="#sec:feas_check_cut_types">feas_check_cut_types</a> </td>
<td>S</td>
<td> -</td>
<td>-</td>
<td>+</td>
<td>+</td>
</tr>
<tr>
<td> <a href="#sec:feas_check_discard_policy">feas_check_discard_policy</a> </td>
<td>S</td>
<td> -</td>
<td>-</td>
<td>+</td>
<td>+</td>
</tr>
<tr>
<td> <a href="#sec:generate_benders_after_so_many_oa">generate_benders_after_so_many_oa</a> </td>
<td>I</td>
<td> -</td>
<td>-</td>
<td>+</td>
<td>+</td>
</tr>
<tr>   <th colspan=9> <a href="#sec:Options_for_feasibility_pump">Options for feasibility pump</a> </th>
</tr>
<tr>
<td> <a href="#sec:fp_log_frequency">fp_log_frequency</a> </td>
<td>F</td>
<td> -</td>
<td>-</td>
<td>+</td>
<td>+</td>
</tr>
<tr>
<td> <a href="#sec:fp_log_level">fp_log_level</a> </td>
<td>I</td>
<td> -</td>
<td>-</td>
<td>+</td>
<td>+</td>
</tr>
<tr>
<td> <a href="#sec:fp_pass_infeasible">fp_pass_infeasible</a> </td>
<td>S</td>
<td> +</td>
<td>+</td>
<td>+</td>
<td>+</td>
</tr>
<tr>   <th colspan=9> <a href="#sec:Options_for_non-convex_problems">Options for non-convex problems</a> </th>
</tr>
<tr>
<td> <a href="#sec:coeff_var_threshold">coeff_var_threshold</a> </td>
<td>F</td>
<td> +</td>
<td>-</td>
<td>-</td>
<td>-</td>
</tr>
<tr>
<td> <a href="#sec:dynamic_def_cutoff_decr">dynamic_def_cutoff_decr</a> </td>
<td>S</td>
<td> +</td>
<td>-</td>
<td>-</td>
<td>-</td>
</tr>
<tr>
<td> <a href="#sec:first_perc_for_cutoff_decr">first_perc_for_cutoff_decr</a> </td>
<td>F</td>
<td> +</td>
<td>-</td>
<td>-</td>
<td>-</td>
</tr>
<tr>
<td> <a href="#sec:max_consecutive_infeasible">max_consecutive_infeasible</a> </td>
<td>I</td>
<td> +</td>
<td>-</td>
<td>-</td>
<td>-</td>
</tr>
<tr>
<td> <a href="#sec:num_resolve_at_infeasibles">num_resolve_at_infeasibles</a> </td>
<td>I</td>
<td> +</td>
<td>-</td>
<td>-</td>
<td>-</td>
</tr>
<tr>
<td> <a href="#sec:num_resolve_at_node">num_resolve_at_node</a> </td>
<td>I</td>
<td> +</td>
<td>-</td>
<td>-</td>
<td>-</td>
</tr>
<tr>
<td> <a href="#sec:num_resolve_at_root">num_resolve_at_root</a> </td>
<td>I</td>
<td> +</td>
<td>-</td>
<td>-</td>
<td>-</td>
</tr>
<tr>
<td> <a href="#sec:second_perc_for_cutoff_decr">second_perc_for_cutoff_decr</a> </td>
<td>F</td>
<td> +</td>
<td>-</td>
<td>-</td>
<td>-</td>
</tr>
<tr>   <th colspan=9> <a href="#sec:Outer_Approximation_cuts_generation">Outer Approximation cuts generation</a> </th>
</tr>
<tr>
<td> <a href="#sec:add_only_violated_oa">add_only_violated_oa</a> </td>
<td>S</td>
<td> -</td>
<td>+</td>
<td>+</td>
<td>+</td>
</tr>
<tr>
<td> <a href="#sec:oa_cuts_log_level">oa_cuts_log_level</a> </td>
<td>I</td>
<td> -</td>
<td>+</td>
<td>+</td>
<td>+</td>
</tr>
<tr>
<td> <a href="#sec:oa_cuts_scope">oa_cuts_scope</a> </td>
<td>S</td>
<td> -</td>
<td>+</td>
<td>+</td>
<td>+</td>
</tr>
<tr>
<td> <a href="#sec:oa_rhs_relax">oa_rhs_relax</a> </td>
<td>F</td>
<td> -</td>
<td>+</td>
<td>+</td>
<td>+</td>
</tr>
<tr>
<td> <a href="#sec:tiny_element">tiny_element</a> </td>
<td>F</td>
<td> -</td>
<td>+</td>
<td>+</td>
<td>+</td>
</tr>
<tr>
<td> <a href="#sec:very_tiny_element">very_tiny_element</a> </td>
<td>F</td>
<td> -</td>
<td>+</td>
<td>+</td>
<td>+</td>
</tr>
<tr>   <th colspan=9> <a href="#sec:Output_and_log-level_options">Output and log-level options</a> </th>
</tr>
<tr>
<td> <a href="#sec:bb_log_interval">bb_log_interval</a> </td>
<td>I</td>
<td> +</td>
<td>+</td>
<td>+</td>
<td>+</td>
</tr>
<tr>
<td> <a href="#sec:bb_log_level">bb_log_level</a> </td>
<td>I</td>
<td> +</td>
<td>+</td>
<td>+</td>
<td>+</td>
</tr>
<tr>
<td> <a href="#sec:lp_log_level">lp_log_level</a> </td>
<td>I</td>
<td> -</td>
<td>+</td>
<td>+</td>
<td>+</td>
</tr>
<tr>
<td> <a href="#sec:nlp_log_at_root">nlp_log_at_root</a> </td>
<td>I</td>
<td> +</td>
<td>+</td>
<td>+</td>
<td>+</td>
</tr>
<tr>   <th colspan=9> <a href="#sec:Strong_branching_setup">Strong branching setup</a> </th>
</tr>
<tr>
<td> <a href="#sec:candidate_sort_criterion">candidate_sort_criterion</a> </td>
<td>S</td>
<td> +</td>
<td>+</td>
<td>+</td>
<td>+</td>
</tr>
<tr>
<td> <a href="#sec:maxmin_crit_have_sol">maxmin_crit_have_sol</a> </td>
<td>F</td>
<td> +</td>
<td>+</td>
<td>+</td>
<td>+</td>
</tr>
<tr>
<td> <a href="#sec:maxmin_crit_no_sol">maxmin_crit_no_sol</a> </td>
<td>F</td>
<td> +</td>
<td>+</td>
<td>+</td>
<td>+</td>
</tr>
<tr>
<td> <a href="#sec:min_number_strong_branch">min_number_strong_branch</a> </td>
<td>I</td>
<td> +</td>
<td>+</td>
<td>+</td>
<td>+</td>
</tr>
<tr>
<td> <a href="#sec:number_before_trust_list">number_before_trust_list</a> </td>
<td>I</td>
<td> +</td>
<td>+</td>
<td>+</td>
<td>+</td>
</tr>
<tr>
<td> <a href="#sec:number_look_ahead">number_look_ahead</a> </td>
<td>I</td>
<td> +</td>
<td>+</td>
<td>+</td>
<td>+</td>
</tr>
<tr>
<td> <a href="#sec:number_strong_branch_root">number_strong_branch_root</a> </td>
<td>I</td>
<td> +</td>
<td>+</td>
<td>+</td>
<td>+</td>
</tr>
<tr>
<td> <a href="#sec:setup_pseudo_frac">setup_pseudo_frac</a> </td>
<td>F</td>
<td> +</td>
<td>+</td>
<td>+</td>
<td>+</td>
</tr>
<tr>
<td> <a href="#sec:trust_strong_branching_for_pseudo_cost">trust_strong_branching_for_pseudo_cost</a> </td>
<td>S</td>
<td> +</td>
<td>+</td>
<td>+</td>
<td>+</td>
</tr>
<tr>   <th colspan=9> <a href="#sec:nlp_interface_option">nlp interface option</a> </th>
</tr>
<tr>
<td> <a href="#sec:file_solution">file_solution</a> </td>
<td>S</td>
<td> +</td>
<td>+</td>
<td>+</td>
<td>+</td>
</tr>
<tr>
<td> <a href="#sec:nlp_log_level">nlp_log_level</a> </td>
<td>I</td>
<td> +</td>
<td>+</td>
<td>+</td>
<td>+</td>
</tr>
<tr>
<td> <a href="#sec:nlp_solver">nlp_solver</a> </td>
<td>S</td>
<td> +</td>
<td>+</td>
<td>+</td>
<td>+</td>
</tr>
<tr>
<td> <a href="#sec:warm_start">warm_start</a> </td>
<td>S</td>
<td> +</td>
<td>-</td>
<td>-</td>
<td>-</td>
</tr>
</tr>
</table>
}
}
\subsection{Algorithm choice}
\label{sec:Algorithm_choice}
\htmlanchor{sec:Algorithm_choice}
\paragraph{algorithm:}\label{sec:algorithm} Choice of the algorithm. $\;$ \\
 This will preset some of the options of bonmin
depending on the algorithm choice.
The default value for this string option is "B-BB".
\\ 
Possible values:
\begin{itemize}
   \item B-BB: simple branch-and-bound algorithm,
   \item B-OA: OA Decomposition algorithm,
   \item B-QG: Quesada and Grossmann branch-and-cut algorithm,
   \item B-Hyb: hybrid outer approximation based branch-and-cut,
   \item B-Ecp: ecp cuts based branch-and-cut a la FilMINT.
   \item B-iFP: Iterated Feasibility Pump for MINLP.
\end{itemize}

\subsection{Branch-and-bound options}
\label{sec:Branch-and-bound_options}
\htmlanchor{sec:Branch-and-bound_options}
\paragraph{allowable\_fraction\_gap:}\label{sec:allowable_fraction_gap} Specify the value of relative gap under which the algorithm stops. $\;$ \\
 Stop the tree search when the gap between the
objective value of the best known solution and
the best bound on the objective of any solution
is less than this fraction of the absolute value
of the best known solution value. The valid range for this real option is 
$-1 \cdot 10^{+20} \le {\tt allowable\_fraction\_gap } \le 1 \cdot 10^{+20}$
and its default value is $0$.


\paragraph{allowable\_gap:}\label{sec:allowable_gap} Specify the value of absolute gap under which the algorithm stops. $\;$ \\
 Stop the tree search when the gap between the
objective value of the best known solution and
the best bound on the objective of any solution
is less than this. The valid range for this real option is 
$-1 \cdot 10^{+20} \le {\tt allowable\_gap } \le 1 \cdot 10^{+20}$
and its default value is $0$.


\paragraph{cutoff:}\label{sec:cutoff} Specify cutoff value. $\;$ \\
 cutoff should be the value of a feasible solution
known by the user (if any). The algorithm will
only look for solutions better than cutoff. The valid range for this real option is 
$-1 \cdot 10^{+100} \le {\tt cutoff } \le 1 \cdot 10^{+100}$
and its default value is $1 \cdot 10^{+100}$.


\paragraph{cutoff\_decr:}\label{sec:cutoff_decr} Specify cutoff decrement. $\;$ \\
 Specify the amount by which cutoff is decremented
below a new best upper-bound (usually a small
positive value but in non-convex problems it may
be a negative value). The valid range for this real option is 
$-1 \cdot 10^{+10} \le {\tt cutoff\_decr } \le 1 \cdot 10^{+10}$
and its default value is $1 \cdot 10^{-05}$.


\paragraph{enable\_dynamic\_nlp:}\label{sec:enable_dynamic_nlp} Enable dynamic linear and quadratic rows addition in nlp $\;$ \\

The default value for this string option is "no".
\\ 
Possible values:
\begin{itemize}
   \item no: 
   \item yes: 
\end{itemize}

\paragraph{integer\_tolerance:}\label{sec:integer_tolerance} Set integer tolerance. $\;$ \\
 Any number within that value of an integer is
considered integer. The valid range for this real option is 
$0 <  {\tt integer\_tolerance } <  {\tt +inf}$
and its default value is $1 \cdot 10^{-06}$.


\paragraph{iteration\_limit:}\label{sec:iteration_limit} Set the cumulated maximum number of iteration in the algorithm used to process nodes continuous relaxations in the branch-and-bound. $\;$ \\
 value 0 deactivates option. The valid range for this integer option is
$0 \le {\tt iteration\_limit } <  {\tt +inf}$
and its default value is $2147483647$.


\paragraph{nlp\_failure\_behavior:}\label{sec:nlp_failure_behavior} Set the behavior when an NLP or a series of NLP are unsolved by Ipopt (we call unsolved an NLP for which Ipopt is not able to guarantee optimality within the specified tolerances). $\;$ \\
 If set to "fathom", the algorithm will fathom the
node when Ipopt fails to find a solution to the
nlp at that node whithin the specified
tolerances. The algorithm then becomes a
heuristic, and the user will be warned that the
solution might not be optimal.
The default value for this string option is "stop".
\\ 
Possible values:
\begin{itemize}
   \item stop: Stop when failure happens.
   \item fathom: Continue when failure happens.
\end{itemize}

\paragraph{node\_comparison:}\label{sec:node_comparison} Choose the node selection strategy. $\;$ \\
 Choose the strategy for selecting the next node
to be processed.
The default value for this string option is "best-bound".
\\ 
Possible values:
\begin{itemize}
   \item best-bound: choose node with the smallest bound,
   \item depth-first: Perform depth first search,
   \item breadth-first: Perform breadth first search,
   \item dynamic: Cbc dynamic strategy (starts with a depth first
search and turn to best bound after 3 integer
feasible solutions have been found).
   \item best-guess: choose node with smallest guessed integer
solution
\end{itemize}

\paragraph{node\_limit:}\label{sec:node_limit} Set the maximum number of nodes explored in the branch-and-bound search. $\;$ \\
 The valid range for this integer option is
$0 \le {\tt node\_limit } <  {\tt +inf}$
and its default value is $2147483647$.


\paragraph{num\_cut\_passes:}\label{sec:num_cut_passes} Set the maximum number of cut passes at regular nodes of the branch-and-cut. $\;$ \\
 The valid range for this integer option is
$0 \le {\tt num\_cut\_passes } <  {\tt +inf}$
and its default value is $1$.


\paragraph{num\_cut\_passes\_at\_root:}\label{sec:num_cut_passes_at_root} Set the maximum number of cut passes at regular nodes of the branch-and-cut. $\;$ \\
 The valid range for this integer option is
$0 \le {\tt num\_cut\_passes\_at\_root } <  {\tt +inf}$
and its default value is $20$.


\paragraph{number\_before\_trust:}\label{sec:number_before_trust} Set the number of branches on a variable before its pseudo costs are to be believed in dynamic strong branching. $\;$ \\
 A value of 0 disables pseudo costs. The valid range for this integer option is
$0 \le {\tt number\_before\_trust } <  {\tt +inf}$
and its default value is $8$.


\paragraph{number\_strong\_branch:}\label{sec:number_strong_branch} Choose the maximum number of variables considered for strong branching. $\;$ \\
 Set the number of variables on which to do strong
branching. The valid range for this integer option is
$0 \le {\tt number\_strong\_branch } <  {\tt +inf}$
and its default value is $20$.


\paragraph{random\_generator\_seed:}\label{sec:random_generator_seed} Set seed for random number generator (a value of -1 sets seeds to time since Epoch). $\;$ \\
 The valid range for this integer option is
$-1 \le {\tt random\_generator\_seed } <  {\tt +inf}$
and its default value is $0$.


\paragraph{read\_solution\_file:}\label{sec:read_solution_file} Read a file with the optimal solution to test if algorithms cuts it. $\;$ \\
 For Debugging purposes only.
The default value for this string option is "no".
\\ 
Possible values:
\begin{itemize}
   \item no: 
   \item yes: 
\end{itemize}

\paragraph{solution\_limit:}\label{sec:solution_limit} Abort after that much integer feasible solution have been found by algorithm $\;$ \\
 value 0 deactivates option The valid range for this integer option is
$0 \le {\tt solution\_limit } <  {\tt +inf}$
and its default value is $2147483647$.


\paragraph{sos\_constraints:}\label{sec:sos_constraints} Wether or not to activate SOS constraints. $\;$ \\
 (only type 1 SOS are supported at the moment)
The default value for this string option is "enable".
\\ 
Possible values:
\begin{itemize}
   \item enable: 
   \item disable: 
\end{itemize}

\paragraph{time\_limit:}\label{sec:time_limit} Set the global maximum computation time (in secs) for the algorithm. $\;$ \\
 The valid range for this real option is 
$0 \le {\tt time\_limit } <  {\tt +inf}$
and its default value is $1 \cdot 10^{+10}$.


\paragraph{tree\_search\_strategy:}\label{sec:tree_search_strategy} Pick a strategy for traversing the tree $\;$ \\
 All strategies can be used in conjunction with
any of the node comparison functions. Options
which affect dfs-dive are max-backtracks-in-dive
and max-dive-depth. The dfs-dive won't work in a
non-convex problem where objective does not
decrease down branches.
The default value for this string option is "probed-dive".
\\ 
Possible values:
\begin{itemize}
   \item top-node:  Always pick the top node as sorted by the node
comparison function
   \item dive: Dive in the tree if possible, otherwise pick
top node as sorted by the tree comparison
function.
   \item probed-dive: Dive in the tree exploring two childs before
continuing the dive at each level.
   \item dfs-dive: Dive in the tree if possible doing a depth
first search. Backtrack on leaves or when a
prescribed depth is attained or when estimate
of best possible integer feasible solution in
subtree is worst than cutoff. Once a prescribed
limit of backtracks is attained pick top node
as sorted by the tree comparison function
   \item dfs-dive-dynamic: Same as dfs-dive but once enough solution are
found switch to best-bound and if too many
nodes switch to depth-first.
\end{itemize}

\paragraph{variable\_selection:}\label{sec:variable_selection} Chooses variable selection strategy $\;$ \\

The default value for this string option is "osi-strong".
\\ 
Possible values:
\begin{itemize}
   \item most-fractional: Choose most fractional variable
   \item strong-branching: Perform strong branching
   \item reliability-branching: Use reliability branching
   \item qp-strong-branching: Perform strong branching with QP approximation
   \item lp-strong-branching: Perform strong branching with LP approximation
   \item nlp-strong-branching: Perform strong branching with NLP approximation
   \item osi-simple: Osi method to do simple branching
   \item osi-strong: Osi method to do strong branching
   \item random: Method to choose branching variable randomly
\end{itemize}

\subsection{MILP cutting planes in hybrid}
\label{sec:MILP_cutting_planes_in_hybrid}
\htmlanchor{sec:MILP_cutting_planes_in_hybrid}
\paragraph{2mir\_cuts:}\label{sec:2mir_cuts} Frequency (in terms of nodes) for generating 2-MIR cuts in branch-and-cut $\;$ \\
 If $k > 0$, cuts are generated every $k$ nodes,
if $-99 < k < 0$ cuts are generated every $-k$
nodes but Cbc may decide to stop generating cuts,
if not enough are generated at the root node, if
$k=-99$ generate cuts only at the root node, if
$k=0$ or $100$ do not generate cuts. The valid range for this integer option is
$-100 \le {\tt 2mir\_cuts } <  {\tt +inf}$
and its default value is $0$.


\paragraph{Gomory\_cuts:}\label{sec:Gomory_cuts} Frequency k (in terms of nodes) for generating Gomory cuts in branch-and-cut. $\;$ \\
 If $k > 0$, cuts are generated every $k$ nodes,
if $-99 < k < 0$ cuts are generated every $-k$
nodes but Cbc may decide to stop generating cuts,
if not enough are generated at the root node, if
$k=-99$ generate cuts only at the root node, if
$k=0$ or $100$ do not generate cuts. The valid range for this integer option is
$-100 \le {\tt Gomory\_cuts } <  {\tt +inf}$
and its default value is $-5$.


\paragraph{clique\_cuts:}\label{sec:clique_cuts} Frequency (in terms of nodes) for generating clique cuts in branch-and-cut $\;$ \\
 If $k > 0$, cuts are generated every $k$ nodes,
if $-99 < k < 0$ cuts are generated every $-k$
nodes but Cbc may decide to stop generating cuts,
if not enough are generated at the root node, if
$k=-99$ generate cuts only at the root node, if
$k=0$ or $100$ do not generate cuts. The valid range for this integer option is
$-100 \le {\tt clique\_cuts } <  {\tt +inf}$
and its default value is $-5$.


\paragraph{cover\_cuts:}\label{sec:cover_cuts} Frequency (in terms of nodes) for generating cover cuts in branch-and-cut $\;$ \\
 If $k > 0$, cuts are generated every $k$ nodes,
if $-99 < k < 0$ cuts are generated every $-k$
nodes but Cbc may decide to stop generating cuts,
if not enough are generated at the root node, if
$k=-99$ generate cuts only at the root node, if
$k=0$ or $100$ do not generate cuts. The valid range for this integer option is
$-100 \le {\tt cover\_cuts } <  {\tt +inf}$
and its default value is $0$.


\paragraph{flow\_cover\_cuts:}\label{sec:flow_cover_cuts} Frequency (in terms of nodes) for generating flow cover cuts in branch-and-cut $\;$ \\
 If $k > 0$, cuts are generated every $k$ nodes,
if $-99 < k < 0$ cuts are generated every $-k$
nodes but Cbc may decide to stop generating cuts,
if not enough are generated at the root node, if
$k=-99$ generate cuts only at the root node, if
$k=0$ or $100$ do not generate cuts. The valid range for this integer option is
$-100 \le {\tt flow\_cover\_cuts } <  {\tt +inf}$
and its default value is $-5$.


\paragraph{lift\_and\_project\_cuts:}\label{sec:lift_and_project_cuts} Frequency (in terms of nodes) for generating lift-and-project cuts in branch-and-cut $\;$ \\
 If $k > 0$, cuts are generated every $k$ nodes,
if $-99 < k < 0$ cuts are generated every $-k$
nodes but Cbc may decide to stop generating cuts,
if not enough are generated at the root node, if
$k=-99$ generate cuts only at the root node, if
$k=0$ or $100$ do not generate cuts. The valid range for this integer option is
$-100 \le {\tt lift\_and\_project\_cuts } <  {\tt +inf}$
and its default value is $0$.


\paragraph{mir\_cuts:}\label{sec:mir_cuts} Frequency (in terms of nodes) for generating MIR cuts in branch-and-cut $\;$ \\
 If $k > 0$, cuts are generated every $k$ nodes,
if $-99 < k < 0$ cuts are generated every $-k$
nodes but Cbc may decide to stop generating cuts,
if not enough are generated at the root node, if
$k=-99$ generate cuts only at the root node, if
$k=0$ or $100$ do not generate cuts. The valid range for this integer option is
$-100 \le {\tt mir\_cuts } <  {\tt +inf}$
and its default value is $-5$.


\paragraph{reduce\_and\_split\_cuts:}\label{sec:reduce_and_split_cuts} Frequency (in terms of nodes) for generating reduce-and-split cuts in branch-and-cut $\;$ \\
 If $k > 0$, cuts are generated every $k$ nodes,
if $-99 < k < 0$ cuts are generated every $-k$
nodes but Cbc may decide to stop generating cuts,
if not enough are generated at the root node, if
$k=-99$ generate cuts only at the root node, if
$k=0$ or $100$ do not generate cuts. The valid range for this integer option is
$-100 \le {\tt reduce\_and\_split\_cuts } <  {\tt +inf}$
and its default value is $0$.


\subsection{MINLP Heuristics}
\label{sec:MINLP_Heuristics}
\htmlanchor{sec:MINLP_Heuristics}
\paragraph{feasibility\_pump\_objective\_norm:}\label{sec:feasibility_pump_objective_norm} Norm of feasibility pump objective function $\;$ \\
 The valid range for this integer option is
$1 \le {\tt feasibility\_pump\_objective\_norm } \le 2$
and its default value is $1$.


\paragraph{heuristic\_RINS:}\label{sec:heuristic_RINS} if yes runs the RINS heuristic $\;$ \\

The default value for this string option is "no".
\\ 
Possible values:
\begin{itemize}
   \item no: don't run it
   \item yes: runs the heuristic
\end{itemize}

\paragraph{heuristic\_dive\_MIP\_fractional:}\label{sec:heuristic_dive_MIP_fractional} if yes runs the Dive MIP Fractional heuristic $\;$ \\

The default value for this string option is "no".
\\ 
Possible values:
\begin{itemize}
   \item no: don't run it
   \item yes: runs the heuristic
\end{itemize}

\paragraph{heuristic\_dive\_MIP\_vectorLength:}\label{sec:heuristic_dive_MIP_vectorLength} if yes runs the Dive MIP VectorLength heuristic $\;$ \\

The default value for this string option is "no".
\\ 
Possible values:
\begin{itemize}
   \item no: don't run it
   \item yes: runs the heuristic
\end{itemize}

\paragraph{heuristic\_dive\_fractional:}\label{sec:heuristic_dive_fractional} if yes runs the Dive Fractional heuristic $\;$ \\

The default value for this string option is "no".
\\ 
Possible values:
\begin{itemize}
   \item no: don't run it
   \item yes: runs the heuristic
\end{itemize}

\paragraph{heuristic\_dive\_vectorLength:}\label{sec:heuristic_dive_vectorLength} if yes runs the Dive VectorLength heuristic $\;$ \\

The default value for this string option is "no".
\\ 
Possible values:
\begin{itemize}
   \item no: don't run it
   \item yes: runs the heuristic
\end{itemize}

\paragraph{heuristic\_feasibility\_pump:}\label{sec:heuristic_feasibility_pump} whether the heuristic feasibility pump should be used $\;$ \\

The default value for this string option is "no".
\\ 
Possible values:
\begin{itemize}
   \item no: don't use it
   \item yes: use it
\end{itemize}

\paragraph{pump\_for\_minlp:}\label{sec:pump_for_minlp} if yes runs FP for MINLP $\;$ \\

The default value for this string option is "no".
\\ 
Possible values:
\begin{itemize}
   \item no: don't run it
   \item yes: runs the heuristic
\end{itemize}

\subsection{Nlp solution robustness}
\label{sec:Nlp_solution_robustness}
\htmlanchor{sec:Nlp_solution_robustness}
\paragraph{max\_consecutive\_failures:}\label{sec:max_consecutive_failures} (temporarily removed) Number $n$ of consecutive unsolved problems before aborting a branch of the tree. $\;$ \\
 When $n > 0$, continue exploring a branch of the
tree until $n$ consecutive problems in the branch
are unsolved (we call unsolved a problem for
which Ipopt can not guarantee optimality within
the specified tolerances). The valid range for this integer option is
$0 \le {\tt max\_consecutive\_failures } <  {\tt +inf}$
and its default value is $10$.


\paragraph{max\_random\_point\_radius:}\label{sec:max_random_point_radius} Set max value r for coordinate of a random point. $\;$ \\
 When picking a random point, coordinate i will be
in the interval [min(max(l,-r),u-r),
max(min(u,r),l+r)] (where l is the lower bound
for the variable and u is its upper bound) The valid range for this real option is 
$0 <  {\tt max\_random\_point\_radius } <  {\tt +inf}$
and its default value is $100000$.


\paragraph{num\_iterations\_suspect:}\label{sec:num_iterations_suspect} Number of iterations over which a node is considered "suspect" (for debugging purposes only, see detailed documentation). $\;$ \\
 When the number of iterations to solve a node is
above this number, the subproblem at this node is
considered to be suspect and it will be outputed
in a file (set to -1 to deactivate this). The valid range for this integer option is
$-1 \le {\tt num\_iterations\_suspect } <  {\tt +inf}$
and its default value is $-1$.


\paragraph{num\_retry\_unsolved\_random\_point:}\label{sec:num_retry_unsolved_random_point} Number $k$ of times that the algorithm will try to resolve an unsolved NLP with a random starting point (we call unsolved an NLP for which Ipopt is not able to guarantee optimality within the specified tolerances). $\;$ \\
 When Ipopt fails to solve a continuous NLP
sub-problem, if $k > 0$, the algorithm will try
again to solve the failed NLP with $k$ new
randomly chosen starting points  or until the
problem is solved with success. The valid range for this integer option is
$0 \le {\tt num\_retry\_unsolved\_random\_point } <  {\tt +inf}$
and its default value is $0$.


\paragraph{random\_point\_perturbation\_interval:}\label{sec:random_point_perturbation_interval} Amount by which starting point is perturbed when choosing to pick random point by perturbating starting point $\;$ \\
 The valid range for this real option is 
$0 <  {\tt random\_point\_perturbation\_interval } <  {\tt +inf}$
and its default value is $1$.


\paragraph{random\_point\_type:}\label{sec:random_point_type} method to choose a random starting point $\;$ \\

The default value for this string option is "Jon".
\\ 
Possible values:
\begin{itemize}
   \item Jon: Choose random point uniformly between the bounds
   \item Andreas: perturb the starting point of the problem
within a prescribed interval
   \item Claudia: perturb the starting point using the
perturbation radius suffix information
\end{itemize}

\paragraph{resolve\_on\_small\_infeasibility:}\label{sec:resolve_on_small_infeasibility} If a locally infeasible problem is infeasible by less than this, resolve it with initial starting point. $\;$ \\
 It is set to 0 by default with Ipopt. For filter
Bonmin sets it to a small value. The valid range for this real option is 
$0 \le {\tt resolve\_on\_small\_infeasibility } <  {\tt +inf}$
and its default value is $0$.


\subsection{Nlp solve options in B-Hyb}
\label{sec:Nlp_solve_options_in_B-Hyb}
\htmlanchor{sec:Nlp_solve_options_in_B-Hyb}
\paragraph{nlp\_solve\_frequency:}\label{sec:nlp_solve_frequency} Specify the frequency (in terms of nodes) at which NLP relaxations are solved in B-Hyb. $\;$ \\
 A frequency of 0 amounts to to never solve the
NLP relaxation. The valid range for this integer option is
$0 \le {\tt nlp\_solve\_frequency } <  {\tt +inf}$
and its default value is $10$.


\paragraph{nlp\_solve\_max\_depth:}\label{sec:nlp_solve_max_depth} Set maximum depth in the tree at which NLP relaxations are solved in B-Hyb. $\;$ \\
 A depth of 0 amounts to to never solve the NLP
relaxation. The valid range for this integer option is
$0 \le {\tt nlp\_solve\_max\_depth } <  {\tt +inf}$
and its default value is $10$.


\paragraph{nlp\_solves\_per\_depth:}\label{sec:nlp_solves_per_depth} Set average number of nodes in the tree at which NLP relaxations are solved in B-Hyb for each depth. $\;$ \\
 The valid range for this real option is 
$0 \le {\tt nlp\_solves\_per\_depth } <  {\tt +inf}$
and its default value is $1 \cdot 10^{+100}$.


\subsection{Options for MILP solver}
\label{sec:Options_for_MILP_solver}
\htmlanchor{sec:Options_for_MILP_solver}
\paragraph{cpx\_parallel\_strategy:}\label{sec:cpx_parallel_strategy} Strategy of parallel search mode in CPLEX. $\;$ \\
 -1 = opportunistic, 0 = automatic, 1 =
deterministic (refer to CPLEX documentation) The valid range for this integer option is
$-1 \le {\tt cpx\_parallel\_strategy } \le 1$
and its default value is $0$.


\paragraph{milp\_log\_level:}\label{sec:milp_log_level} specify MILP solver log level. $\;$ \\
 Set the level of output of the MILP subsolver in
OA : 0 - none, 1 - minimal, 2 - normal low, 3 -
normal high The valid range for this integer option is
$0 \le {\tt milp\_log\_level } \le 4$
and its default value is $0$.


\paragraph{milp\_solver:}\label{sec:milp_solver} Choose the subsolver to solve MILP sub-problems in OA decompositions. $\;$ \\
  To use Cplex, a valid license is required and
you should have compiled OsiCpx in COIN-OR  (see
Osi documentation).
The default value for this string option is "Cbc\_D".
\\ 
Possible values:
\begin{itemize}
   \item Cbc\_D: Coin Branch and Cut with its default
   \item Cbc\_Par: Coin Branch and Cut with passed parameters
   \item Cplex: Ilog Cplex
\end{itemize}

\paragraph{milp\_strategy:}\label{sec:milp_strategy} Choose a strategy for MILPs. $\;$ \\

The default value for this string option is "find\_good\_sol".
\\ 
Possible values:
\begin{itemize}
   \item find\_good\_sol: Stop sub milps when a solution improving the
incumbent is found
   \item solve\_to\_optimality: Solve MILPs to optimality
\end{itemize}

\paragraph{number\_cpx\_threads:}\label{sec:number_cpx_threads} Set number of threads to use with cplex. $\;$ \\
 (refer to CPLEX documentation) The valid range for this integer option is
$0 \le {\tt number\_cpx\_threads } <  {\tt +inf}$
and its default value is $0$.


\subsection{Options for OA decomposition}
\label{sec:Options_for_OA_decomposition}
\htmlanchor{sec:Options_for_OA_decomposition}
\paragraph{oa\_decomposition:}\label{sec:oa_decomposition} If yes do initial OA decomposition $\;$ \\

The default value for this string option is "no".
\\ 
Possible values:
\begin{itemize}
   \item no: 
   \item yes: 
\end{itemize}

\paragraph{oa\_log\_frequency:}\label{sec:oa_log_frequency} display an update on lower and upper bounds in OA every n seconds $\;$ \\
 The valid range for this real option is 
$0 <  {\tt oa\_log\_frequency } <  {\tt +inf}$
and its default value is $100$.


\paragraph{oa\_log\_level:}\label{sec:oa_log_level} specify OA iterations log level. $\;$ \\
 Set the level of output of OA decomposition
solver : 0 - none, 1 - normal, 2 - verbose The valid range for this integer option is
$0 \le {\tt oa\_log\_level } \le 2$
and its default value is $1$.


\subsection{Options for ecp cuts generation}
\label{sec:Options_for_ecp_cuts_generation}
\htmlanchor{sec:Options_for_ecp_cuts_generation}
\paragraph{ecp\_abs\_tol:}\label{sec:ecp_abs_tol} Set the absolute termination tolerance for ECP rounds. $\;$ \\
 The valid range for this real option is 
$0 \le {\tt ecp\_abs\_tol } <  {\tt +inf}$
and its default value is $1 \cdot 10^{-06}$.


\paragraph{ecp\_max\_rounds:}\label{sec:ecp_max_rounds} Set the maximal number of rounds of ECP cuts. $\;$ \\
 The valid range for this integer option is
$0 \le {\tt ecp\_max\_rounds } <  {\tt +inf}$
and its default value is $5$.


\paragraph{ecp\_probability\_factor:}\label{sec:ecp_probability_factor} Factor appearing in formula for skipping ECP cuts. $\;$ \\
 Choosing -1 disables the skipping. The valid range for this real option is 
${\tt -inf} <  {\tt ecp\_probability\_factor } <  {\tt +inf}$
and its default value is $10$.


\paragraph{ecp\_rel\_tol:}\label{sec:ecp_rel_tol} Set the relative termination tolerance for ECP rounds. $\;$ \\
 The valid range for this real option is 
$0 \le {\tt ecp\_rel\_tol } <  {\tt +inf}$
and its default value is $0$.


\paragraph{filmint\_ecp\_cuts:}\label{sec:filmint_ecp_cuts} Specify the frequency (in terms of nodes) at which some a la filmint ecp cuts are generated. $\;$ \\
 A frequency of 0 amounts to to never solve the
NLP relaxation. The valid range for this integer option is
$0 \le {\tt filmint\_ecp\_cuts } <  {\tt +inf}$
and its default value is $0$.


\subsection{Options for feasibility checker using OA cuts}
\label{sec:Options_for_feasibility_checker_using_OA_cuts}
\htmlanchor{sec:Options_for_feasibility_checker_using_OA_cuts}
\paragraph{feas\_check\_cut\_types:}\label{sec:feas_check_cut_types} Choose the type of cuts generated when an integer feasible solution is found $\;$ \\
 If it seems too much memory is used should try
Benders to use less
The default value for this string option is "outer-approx".
\\ 
Possible values:
\begin{itemize}
   \item outer-approx: Generate a set of Outer Approximations cuts.
   \item Benders: Generate a single Benders cut.
\end{itemize}

\paragraph{feas\_check\_discard\_policy:}\label{sec:feas_check_discard_policy} How cuts from feasibility checker are discarded $\;$ \\
 Normally to avoid cycle cuts from feasibility
checker should not be discarded in the node where
they are generated. However Cbc sometimes does it
if no care is taken which can lead to an infinite
loop in Bonmin (usualy on simple problems). To
avoid this one can instruct Cbc to never discard
a cut but if we do that for all cuts it can lead
to memory problems. The default policy here is to
detect cycles and only then impose to Cbc to keep
the cut. The two other alternative are to
instruct Cbc to keep all cuts or to just ignore
the problem and hope for the best
The default value for this string option is "detect-cycles".
\\ 
Possible values:
\begin{itemize}
   \item detect-cycles: Detect if a cycle occurs and only in this case
force not to discard.
   \item keep-all: Force cuts from feasibility checker not to be
discarded (memory hungry but sometimes better).
   \item treated-as-normal: Cuts from memory checker can be discarded as
any other cuts (code may cycle then)
\end{itemize}

\paragraph{generate\_benders\_after\_so\_many\_oa:}\label{sec:generate_benders_after_so_many_oa} Specify that after so many oa cuts have been generated Benders cuts should be generated instead. $\;$ \\
 It seems that sometimes generating too many oa
cuts slows down the optimization compared to
Benders due to the size of the LP. With this
option we specify that after so many OA cuts have
been generated we should switch to Benders cuts. The valid range for this integer option is
$0 \le {\tt generate\_benders\_after\_so\_many\_oa } <  {\tt +inf}$
and its default value is $5000$.


\subsection{Options for feasibility pump}
\label{sec:Options_for_feasibility_pump}
\htmlanchor{sec:Options_for_feasibility_pump}
\paragraph{fp\_log\_frequency:}\label{sec:fp_log_frequency} display an update on lower and upper bounds in FP every n seconds $\;$ \\
 The valid range for this real option is 
$0 <  {\tt fp\_log\_frequency } <  {\tt +inf}$
and its default value is $100$.


\paragraph{fp\_log\_level:}\label{sec:fp_log_level} specify FP iterations log level. $\;$ \\
 Set the level of output of OA decomposition
solver : 0 - none, 1 - normal, 2 - verbose The valid range for this integer option is
$0 \le {\tt fp\_log\_level } \le 2$
and its default value is $1$.


\paragraph{fp\_pass\_infeasible:}\label{sec:fp_pass_infeasible} Say whether feasibility pump should claim to converge or not $\;$ \\

The default value for this string option is "no".
\\ 
Possible values:
\begin{itemize}
   \item no: When master MILP is infeasible just bail out
(don't stop all algorithm). This is the option
for using in B-Hyb.
   \item yes: Claim convergence, numerically dangerous.
\end{itemize}

\subsection{Options for non-convex problems}
\label{sec:Options_for_non-convex_problems}
\htmlanchor{sec:Options_for_non-convex_problems}
\paragraph{coeff\_var\_threshold:}\label{sec:coeff_var_threshold} Coefficient of variation threshold (for dynamic definition of cutoff\_decr). $\;$ \\
 The valid range for this real option is 
$0 \le {\tt coeff\_var\_threshold } <  {\tt +inf}$
and its default value is $0.1$.


\paragraph{dynamic\_def\_cutoff\_decr:}\label{sec:dynamic_def_cutoff_decr} Do you want to define the parameter cutoff\_decr dynamically? $\;$ \\

The default value for this string option is "no".
\\ 
Possible values:
\begin{itemize}
   \item no: No, define it statically
   \item yes: Yes, define it dynamically
\end{itemize}

\paragraph{first\_perc\_for\_cutoff\_decr:}\label{sec:first_perc_for_cutoff_decr} The percentage used when, the coeff of variance is smaller than the threshold, to compute the cutoff\_decr dynamically. $\;$ \\
 The valid range for this real option is 
${\tt -inf} <  {\tt first\_perc\_for\_cutoff\_decr } <  {\tt +inf}$
and its default value is $-0.02$.


\paragraph{max\_consecutive\_infeasible:}\label{sec:max_consecutive_infeasible} Number of consecutive infeasible subproblems before aborting a branch. $\;$ \\
 Will continue exploring a branch of the tree
until "max\_consecutive\_infeasible"consecutive
problems are infeasibles by the NLP sub-solver. The valid range for this integer option is
$0 \le {\tt max\_consecutive\_infeasible } <  {\tt +inf}$
and its default value is $0$.


\paragraph{num\_resolve\_at\_infeasibles:}\label{sec:num_resolve_at_infeasibles} Number $k$ of tries to resolve an infeasible node (other than the root) of the tree with different starting point. $\;$ \\
 The algorithm will solve all the infeasible nodes
with $k$ different random starting points and
will keep the best local optimum found. The valid range for this integer option is
$0 \le {\tt num\_resolve\_at\_infeasibles } <  {\tt +inf}$
and its default value is $0$.


\paragraph{num\_resolve\_at\_node:}\label{sec:num_resolve_at_node} Number $k$ of tries to resolve a node (other than the root) of the tree with different starting point. $\;$ \\
 The algorithm will solve all the nodes with $k$
different random starting points and will keep
the best local optimum found. The valid range for this integer option is
$0 \le {\tt num\_resolve\_at\_node } <  {\tt +inf}$
and its default value is $0$.


\paragraph{num\_resolve\_at\_root:}\label{sec:num_resolve_at_root} Number $k$ of tries to resolve the root node with different starting points. $\;$ \\
 The algorithm will solve the root node with $k$
random starting points and will keep the best
local optimum found. The valid range for this integer option is
$0 \le {\tt num\_resolve\_at\_root } <  {\tt +inf}$
and its default value is $0$.


\paragraph{second\_perc\_for\_cutoff\_decr:}\label{sec:second_perc_for_cutoff_decr} The percentage used when, the coeff of variance is greater than the threshold, to compute the cutoff\_decr dynamically. $\;$ \\
 The valid range for this real option is 
${\tt -inf} <  {\tt second\_perc\_for\_cutoff\_decr } <  {\tt +inf}$
and its default value is $-0.05$.


\subsection{Outer Approximation cuts generation}
\label{sec:Outer_Approximation_cuts_generation}
\htmlanchor{sec:Outer_Approximation_cuts_generation}
\paragraph{add\_only\_violated\_oa:}\label{sec:add_only_violated_oa} Do we add all OA cuts or only the ones violated by current point? $\;$ \\

The default value for this string option is "no".
\\ 
Possible values:
\begin{itemize}
   \item no: Add all cuts
   \item yes: Add only violated Cuts
\end{itemize}

\paragraph{oa\_cuts\_log\_level:}\label{sec:oa_cuts_log_level} level of log when generating OA cuts. $\;$ \\
 0: outputs nothing,
1: when a cut is generated,
its violation and index of row from which it
originates,
2: always output violation of the
cut.
3: output generated cuts incidence vectors. The valid range for this integer option is
$0 \le {\tt oa\_cuts\_log\_level } <  {\tt +inf}$
and its default value is $0$.


\paragraph{oa\_cuts\_scope:}\label{sec:oa_cuts_scope} Specify if OA cuts added are to be set globally or locally valid $\;$ \\

The default value for this string option is "global".
\\ 
Possible values:
\begin{itemize}
   \item local: Cuts are treated as locally valid
   \item global: Cuts are treated as globally valid
\end{itemize}

\paragraph{oa\_rhs\_relax:}\label{sec:oa_rhs_relax} Value by which to relax OA cut $\;$ \\
 RHS of OA constraints will be relaxed by this
amount times the absolute value of the initial
rhs if it is >= 1 (otherwise by this amount). The valid range for this real option is 
$-0 \le {\tt oa\_rhs\_relax } <  {\tt +inf}$
and its default value is $1 \cdot 10^{-08}$.


\paragraph{tiny\_element:}\label{sec:tiny_element} Value for tiny element in OA cut $\;$ \\
 We will remove "cleanly" (by relaxing cut) an
element lower than this. The valid range for this real option is 
$-0 \le {\tt tiny\_element } <  {\tt +inf}$
and its default value is $1 \cdot 10^{-08}$.


\paragraph{very\_tiny\_element:}\label{sec:very_tiny_element} Value for very tiny element in OA cut $\;$ \\
 Algorithm will take the risk of neglecting an
element lower than this. The valid range for this real option is 
$-0 \le {\tt very\_tiny\_element } <  {\tt +inf}$
and its default value is $1 \cdot 10^{-17}$.


\subsection{Output and log-level options}
\label{sec:Output_and_log-level_options}
\htmlanchor{sec:Output_and_log-level_options}
\paragraph{bb\_log\_interval:}\label{sec:bb_log_interval} Interval at which node level output is printed. $\;$ \\
 Set the interval (in terms of number of nodes) at
which a log on node resolutions (consisting of
lower and upper bounds) is given. The valid range for this integer option is
$0 \le {\tt bb\_log\_interval } <  {\tt +inf}$
and its default value is $100$.


\paragraph{bb\_log\_level:}\label{sec:bb_log_level} specify main branch-and-bound log level. $\;$ \\
 Set the level of output of the branch-and-bound :
0 - none, 1 - minimal, 2 - normal low, 3 - normal
high The valid range for this integer option is
$0 \le {\tt bb\_log\_level } \le 5$
and its default value is $1$.


\paragraph{lp\_log\_level:}\label{sec:lp_log_level} specify LP log level. $\;$ \\
 Set the level of output of the linear programming
sub-solver in B-Hyb or B-QG : 0 - none, 1 -
minimal, 2 - normal low, 3 - normal high, 4 -
verbose The valid range for this integer option is
$0 \le {\tt lp\_log\_level } \le 4$
and its default value is $0$.


\paragraph{nlp\_log\_at\_root:}\label{sec:nlp_log_at_root}  Specify a different log level for root relaxtion. $\;$ \\
 The valid range for this integer option is
$0 \le {\tt nlp\_log\_at\_root } \le 12$
and its default value is $0$.


\subsection{Strong branching setup}
\label{sec:Strong_branching_setup}
\htmlanchor{sec:Strong_branching_setup}
\paragraph{candidate\_sort\_criterion:}\label{sec:candidate_sort_criterion} Choice of the criterion to choose candidates in strong-branching $\;$ \\

The default value for this string option is "best-ps-cost".
\\ 
Possible values:
\begin{itemize}
   \item best-ps-cost: Sort by decreasing pseudo-cost
   \item worst-ps-cost: Sort by increasing pseudo-cost
   \item most-fractional: Sort by decreasing integer infeasibility
   \item least-fractional: Sort by increasing integer infeasibility
\end{itemize}

\paragraph{maxmin\_crit\_have\_sol:}\label{sec:maxmin_crit_have_sol} Weight towards minimum in of lower and upper branching estimates when a solution has been found. $\;$ \\
 The valid range for this real option is 
$0 \le {\tt maxmin\_crit\_have\_sol } \le 1$
and its default value is $0.1$.


\paragraph{maxmin\_crit\_no\_sol:}\label{sec:maxmin_crit_no_sol} Weight towards minimum in of lower and upper branching estimates when no solution has been found yet. $\;$ \\
 The valid range for this real option is 
$0 \le {\tt maxmin\_crit\_no\_sol } \le 1$
and its default value is $0.7$.


\paragraph{min\_number\_strong\_branch:}\label{sec:min_number_strong_branch} Sets minimum number of variables for strong branching (overriding trust) $\;$ \\
 The valid range for this integer option is
$0 \le {\tt min\_number\_strong\_branch } <  {\tt +inf}$
and its default value is $0$.


\paragraph{number\_before\_trust\_list:}\label{sec:number_before_trust_list} Set the number of branches on a variable before its pseudo costs are to be believed during setup of strong branching candidate list. $\;$ \\
 The default value is that of
"number\_before\_trust" The valid range for this integer option is
$-1 \le {\tt number\_before\_trust\_list } <  {\tt +inf}$
and its default value is $0$.


\paragraph{number\_look\_ahead:}\label{sec:number_look_ahead} Sets limit of look-ahead strong-branching trials $\;$ \\
 The valid range for this integer option is
$0 \le {\tt number\_look\_ahead } <  {\tt +inf}$
and its default value is $0$.


\paragraph{number\_strong\_branch\_root:}\label{sec:number_strong_branch_root} Maximum number of variables considered for strong branching in root node. $\;$ \\
 The valid range for this integer option is
$0 \le {\tt number\_strong\_branch\_root } <  {\tt +inf}$
and its default value is $2147483647$.


\paragraph{setup\_pseudo\_frac:}\label{sec:setup_pseudo_frac} Proportion of strong branching list that has to be taken from most-integer-infeasible list. $\;$ \\
 The valid range for this real option is 
$0 \le {\tt setup\_pseudo\_frac } \le 1$
and its default value is $0.5$.


\paragraph{trust\_strong\_branching\_for\_pseudo\_cost:}\label{sec:trust_strong_branching_for_pseudo_cost} Whether or not to trust strong branching results for updating pseudo costs. $\;$ \\

The default value for this string option is "yes".
\\ 
Possible values:
\begin{itemize}
   \item no: 
   \item yes: 
\end{itemize}

\subsection{nlp interface option}
\label{sec:nlp_interface_option}
\htmlanchor{sec:nlp_interface_option}
\paragraph{file\_solution:}\label{sec:file_solution} Write a file bonmin.sol with the solution $\;$ \\

The default value for this string option is "no".
\\ 
Possible values:
\begin{itemize}
   \item yes: 
   \item no: 
\end{itemize}

\paragraph{nlp\_log\_level:}\label{sec:nlp_log_level} specify NLP solver interface log level (independent from ipopt print\_level). $\;$ \\
 Set the level of output of the OsiTMINLPInterface
: 0 - none, 1 - normal, 2 - verbose The valid range for this integer option is
$0 \le {\tt nlp\_log\_level } \le 2$
and its default value is $1$.


\paragraph{nlp\_solver:}\label{sec:nlp_solver} Choice of the solver for local optima of continuous nlp's $\;$ \\
 Note that option will work only if the specified
solver has been installed. Ipopt will usualy be
installed with Bonmin by default. For FilterSQP
please see
http://www-unix.mcs.anl.gov/~leyffer/solvers.html
on how to obtain it and
https://projects.coin-or.org/Bonmin/wiki/HintTric-
ks on how to configure Bonmin to use it.
The default value for this string option is "Ipopt".
\\ 
Possible values:
\begin{itemize}
   \item Ipopt: Interior Point OPTimizer
(https://projects.coin-or.org/Ipopt)
   \item filterSQP: Sequential quadratic programming trust region
algorithm
(http://www-unix.mcs.anl.gov/~leyffer/solvers.h-
tml)
   \item all: run all available solvers at each node
\end{itemize}

\paragraph{warm\_start:}\label{sec:warm_start} Select the warm start method $\;$ \\
 This will affect the function getWarmStart(), and
as a consequence the warm starting in the various
algorithms.
The default value for this string option is "none".
\\ 
Possible values:
\begin{itemize}
   \item none: No warm start, just start NLPs from optimal
solution of the root relaxation
   \item fake\_basis: builds fake basis, useful for cut management in
Cbc (warm start is the same as in none)
   \item optimum: Warm start with direct parent optimum
   \item interior\_point: Warm start with an interior point of direct
parent
\end{itemize}


}


\end{document}
