\latexhtml{}{
\HCode{
<table border="1">
<tr>
<td>Option </td>
<td> type </td>
<td> B-BB</td>
<td> B-OA</td>
<td> B-QG</td>
<td> B-Hyb</td>
</tr>
<tr>   <th colspan=9> <a href="#sec:Algorithmchoice">Algorithm choice</a> </th>
</tr>
<tr>
<td>algorithm</td>
<td>S</td>
<td> +</td>
<td>+</td>
<td>+</td>
<td>+</td>
</tr>
<tr>   <th colspan=9> <a href="#sec:Branch-and-boundoptions">Branch-and-bound options</a> </th>
</tr>
<tr>
<td>allowable_fraction_gap</td>
<td>F</td>
<td> +</td>
<td>+</td>
<td>+</td>
<td>+</td>
</tr>
<tr>
<td>allowable_gap</td>
<td>F</td>
<td> +</td>
<td>+</td>
<td>+</td>
<td>+</td>
</tr>
<tr>
<td>cutoff</td>
<td>F</td>
<td> +</td>
<td>+</td>
<td>+</td>
<td>+</td>
</tr>
<tr>
<td>cutoff_decr</td>
<td>F</td>
<td> +</td>
<td>+</td>
<td>+</td>
<td>+</td>
</tr>
<tr>
<td>enable_dynamic_nlp</td>
<td>S</td>
<td> +</td>
<td>-</td>
<td>-</td>
<td>-</td>
</tr>
<tr>
<td>integer_tolerance</td>
<td>F</td>
<td> +</td>
<td>+</td>
<td>+</td>
<td>+</td>
</tr>
<tr>
<td>iteration_limit</td>
<td>I</td>
<td> +</td>
<td>+</td>
<td>+</td>
<td>+</td>
</tr>
<tr>
<td>nlp_failure_behavior</td>
<td>S</td>
<td> +</td>
<td>-</td>
<td>-</td>
<td>-</td>
</tr>
<tr>
<td>node_comparison</td>
<td>S</td>
<td> +</td>
<td>+</td>
<td>+</td>
<td>+</td>
</tr>
<tr>
<td>node_limit</td>
<td>I</td>
<td> +</td>
<td>+</td>
<td>+</td>
<td>+</td>
</tr>
<tr>
<td>num_cut_passes</td>
<td>I</td>
<td> -</td>
<td>-</td>
<td>+</td>
<td>+</td>
</tr>
<tr>
<td>num_cut_passes_at_root</td>
<td>I</td>
<td> -</td>
<td>-</td>
<td>+</td>
<td>+</td>
</tr>
<tr>
<td>number_before_trust</td>
<td>I</td>
<td> +</td>
<td>+</td>
<td>+</td>
<td>+</td>
</tr>
<tr>
<td>number_strong_branch</td>
<td>I</td>
<td> +</td>
<td>+</td>
<td>+</td>
<td>+</td>
</tr>
<tr>
<td>random_generator_seed</td>
<td>I</td>
<td> +</td>
<td>+</td>
<td>+</td>
<td>+</td>
</tr>
<tr>
<td>read_solution_file</td>
<td>S</td>
<td> +</td>
<td>+</td>
<td>+</td>
<td>+</td>
</tr>
<tr>
<td>solution_limit</td>
<td>I</td>
<td> +</td>
<td>+</td>
<td>+</td>
<td>+</td>
</tr>
<tr>
<td>sos_constraints</td>
<td>S</td>
<td> +</td>
<td>+</td>
<td>+</td>
<td>+</td>
</tr>
<tr>
<td>time_limit</td>
<td>F</td>
<td> +</td>
<td>+</td>
<td>+</td>
<td>+</td>
</tr>
<tr>
<td>tree_search_strategy</td>
<td>S</td>
<td> +</td>
<td>+</td>
<td>+</td>
<td>+</td>
</tr>
<tr>
<td>variable_selection</td>
<td>S</td>
<td> +</td>
<td>-</td>
<td>+</td>
<td>+</td>
</tr>
<tr>   <th colspan=9> <a href="#sec:ECPcutsgeneration">ECP cuts generation</a> </th>
</tr>
<tr>
<td>ecp_abs_tol</td>
<td>F</td>
<td> -</td>
<td>-</td>
<td>+</td>
<td>+</td>
</tr>
<tr>
<td>ecp_max_rounds</td>
<td>I</td>
<td> -</td>
<td>-</td>
<td>+</td>
<td>+</td>
</tr>
<tr>
<td>ecp_probability_factor</td>
<td>F</td>
<td> -</td>
<td>-</td>
<td>+</td>
<td>+</td>
</tr>
<tr>
<td>ecp_rel_tol</td>
<td>F</td>
<td> -</td>
<td>-</td>
<td>+</td>
<td>+</td>
</tr>
<tr>
<td>filmint_ecp_cuts</td>
<td>I</td>
<td> -</td>
<td>-</td>
<td>+</td>
<td>+</td>
</tr>
<tr>   <th colspan=9> <a href="#sec:FeasibilitycheckerusingOAcuts">Feasibility checker using OA cuts</a> </th>
</tr>
<tr>
<td>feas_check_cut_types</td>
<td>S</td>
<td> -</td>
<td>-</td>
<td>+</td>
<td>+</td>
</tr>
<tr>
<td>feas_check_discard_policy</td>
<td>S</td>
<td> -</td>
<td>-</td>
<td>+</td>
<td>+</td>
</tr>
<tr>
<td>generate_benders_after_so_many_oa</td>
<td>I</td>
<td> -</td>
<td>-</td>
<td>+</td>
<td>+</td>
</tr>
<tr>   <th colspan=9> <a href="#sec:MILPSolver">MILP Solver</a> </th>
</tr>
<tr>
<td>cpx_parallel_strategy</td>
<td>I</td>
<td> -</td>
<td>-</td>
<td>-</td>
<td>-</td>
</tr>
<tr>
<td>milp_solver</td>
<td>S</td>
<td> -</td>
<td>-</td>
<td>-</td>
<td>-</td>
</tr>
<tr>
<td>milp_strategy</td>
<td>S</td>
<td> -</td>
<td>-</td>
<td>-</td>
<td>-</td>
</tr>
<tr>
<td>number_cpx_threads</td>
<td>I</td>
<td> -</td>
<td>-</td>
<td>-</td>
<td>-</td>
</tr>
<tr>   <th colspan=9> <a href="#sec:MILPcuttingplanesinhybridalgorithm">MILP cutting planes in hybrid algorithm</a> </th>
</tr>
<tr>
<td>2mir_cuts</td>
<td>I</td>
<td> -</td>
<td>+</td>
<td>+</td>
<td>+</td>
</tr>
<tr>
<td>Gomory_cuts</td>
<td>I</td>
<td> -</td>
<td>+</td>
<td>+</td>
<td>+</td>
</tr>
<tr>
<td>clique_cuts</td>
<td>I</td>
<td> -</td>
<td>+</td>
<td>+</td>
<td>+</td>
</tr>
<tr>
<td>cover_cuts</td>
<td>I</td>
<td> -</td>
<td>+</td>
<td>+</td>
<td>+</td>
</tr>
<tr>
<td>flow_cover_cuts</td>
<td>I</td>
<td> -</td>
<td>+</td>
<td>+</td>
<td>+</td>
</tr>
<tr>
<td>lift_and_project_cuts</td>
<td>I</td>
<td> -</td>
<td>+</td>
<td>+</td>
<td>+</td>
</tr>
<tr>
<td>mir_cuts</td>
<td>I</td>
<td> -</td>
<td>+</td>
<td>+</td>
<td>+</td>
</tr>
<tr>
<td>reduce_and_split_cuts</td>
<td>I</td>
<td> -</td>
<td>+</td>
<td>+</td>
<td>+</td>
</tr>
<tr>   <th colspan=9> <a href="#sec:NLPinterface">NLP interface</a> </th>
</tr>
<tr>
<td>nlp_solver</td>
<td>S</td>
<td> +</td>
<td>+</td>
<td>+</td>
<td>+</td>
</tr>
<tr>
<td>warm_start</td>
<td>S</td>
<td> +</td>
<td>-</td>
<td>-</td>
<td>-</td>
</tr>
<tr>   <th colspan=9> <a href="#sec:NLPsolutionrobustness">NLP solution robustness</a> </th>
</tr>
<tr>
<td>max_consecutive_failures</td>
<td>I</td>
<td> +</td>
<td>-</td>
<td>-</td>
<td>-</td>
</tr>
<tr>
<td>max_random_point_radius</td>
<td>F</td>
<td> +</td>
<td>-</td>
<td>-</td>
<td>-</td>
</tr>
<tr>
<td>num_iterations_suspect</td>
<td>I</td>
<td> +</td>
<td>+</td>
<td>+</td>
<td>+</td>
</tr>
<tr>
<td>num_retry_unsolved_random_point</td>
<td>I</td>
<td> +</td>
<td>+</td>
<td>+</td>
<td>+</td>
</tr>
<tr>
<td>random_point_perturbation_interval</td>
<td>F</td>
<td> +</td>
<td>-</td>
<td>-</td>
<td>-</td>
</tr>
<tr>
<td>random_point_type</td>
<td>S</td>
<td> +</td>
<td>-</td>
<td>-</td>
<td>-</td>
</tr>
<tr>
<td>resolve_on_small_infeasibility</td>
<td>F</td>
<td> +</td>
<td>-</td>
<td>-</td>
<td>-</td>
</tr>
<tr>   <th colspan=9> <a href="#sec:NLPsolvesinhybridalgorithm(B-Hyb)">NLP solves in hybrid algorithm (B-Hyb)</a> </th>
</tr>
<tr>
<td>nlp_solve_frequency</td>
<td>I</td>
<td> -</td>
<td>-</td>
<td>-</td>
<td>+</td>
</tr>
<tr>
<td>nlp_solve_max_depth</td>
<td>I</td>
<td> -</td>
<td>-</td>
<td>-</td>
<td>+</td>
</tr>
<tr>
<td>nlp_solves_per_depth</td>
<td>F</td>
<td> -</td>
<td>-</td>
<td>-</td>
<td>+</td>
</tr>
<tr>   <th colspan=9> <a href="#sec:Nonconvexproblems">Nonconvex problems</a> </th>
</tr>
<tr>
<td>coeff_var_threshold</td>
<td>F</td>
<td> +</td>
<td>-</td>
<td>-</td>
<td>-</td>
</tr>
<tr>
<td>dynamic_def_cutoff_decr</td>
<td>S</td>
<td> +</td>
<td>-</td>
<td>-</td>
<td>-</td>
</tr>
<tr>
<td>first_perc_for_cutoff_decr</td>
<td>F</td>
<td> +</td>
<td>-</td>
<td>-</td>
<td>-</td>
</tr>
<tr>
<td>max_consecutive_infeasible</td>
<td>I</td>
<td> +</td>
<td>-</td>
<td>-</td>
<td>-</td>
</tr>
<tr>
<td>num_resolve_at_infeasibles</td>
<td>I</td>
<td> +</td>
<td>-</td>
<td>-</td>
<td>-</td>
</tr>
<tr>
<td>num_resolve_at_node</td>
<td>I</td>
<td> +</td>
<td>-</td>
<td>-</td>
<td>-</td>
</tr>
<tr>
<td>num_resolve_at_root</td>
<td>I</td>
<td> +</td>
<td>-</td>
<td>-</td>
<td>-</td>
</tr>
<tr>
<td>second_perc_for_cutoff_decr</td>
<td>F</td>
<td> +</td>
<td>-</td>
<td>-</td>
<td>-</td>
</tr>
<tr>   <th colspan=9> <a href="#sec:OuterApproximationDecomposition(B-OA)">Outer Approximation Decomposition (B-OA)</a> </th>
</tr>
<tr>
<td>oa_decomposition</td>
<td>S</td>
<td> -</td>
<td>-</td>
<td>+</td>
<td>+</td>
</tr>
<tr>   <th colspan=9> <a href="#sec:OuterApproximationcutsgeneration">Outer Approximation cuts generation</a> </th>
</tr>
<tr>
<td>add_only_violated_oa</td>
<td>S</td>
<td> -</td>
<td>+</td>
<td>+</td>
<td>+</td>
</tr>
<tr>
<td>oa_cuts_scope</td>
<td>S</td>
<td> -</td>
<td>+</td>
<td>+</td>
<td>+</td>
</tr>
<tr>
<td>oa_rhs_relax</td>
<td>F</td>
<td> -</td>
<td>+</td>
<td>+</td>
<td>+</td>
</tr>
<tr>
<td>tiny_element</td>
<td>F</td>
<td> -</td>
<td>+</td>
<td>+</td>
<td>+</td>
</tr>
<tr>
<td>very_tiny_element</td>
<td>F</td>
<td> -</td>
<td>+</td>
<td>+</td>
<td>+</td>
</tr>
<tr>   <th colspan=9> <a href="#sec:OutputandLoglevel">Output and Loglevel</a> </th>
</tr>
<tr>
<td>bb_log_interval</td>
<td>I</td>
<td> +</td>
<td>+</td>
<td>+</td>
<td>+</td>
</tr>
<tr>
<td>bb_log_level</td>
<td>I</td>
<td> +</td>
<td>+</td>
<td>+</td>
<td>+</td>
</tr>
<tr>
<td>file_solution</td>
<td>S</td>
<td> +</td>
<td>+</td>
<td>+</td>
<td>+</td>
</tr>
<tr>
<td>fp_log_frequency</td>
<td>F</td>
<td> -</td>
<td>-</td>
<td>+</td>
<td>+</td>
</tr>
<tr>
<td>fp_log_level</td>
<td>I</td>
<td> -</td>
<td>-</td>
<td>+</td>
<td>+</td>
</tr>
<tr>
<td>lp_log_level</td>
<td>I</td>
<td> -</td>
<td>+</td>
<td>+</td>
<td>+</td>
</tr>
<tr>
<td>milp_log_level</td>
<td>I</td>
<td> -</td>
<td>-</td>
<td>-</td>
<td>-</td>
</tr>
<tr>
<td>nlp_log_at_root</td>
<td>I</td>
<td> +</td>
<td>+</td>
<td>+</td>
<td>+</td>
</tr>
<tr>
<td>nlp_log_level</td>
<td>I</td>
<td> +</td>
<td>+</td>
<td>+</td>
<td>+</td>
</tr>
<tr>
<td>oa_cuts_log_level</td>
<td>I</td>
<td> -</td>
<td>+</td>
<td>+</td>
<td>+</td>
</tr>
<tr>
<td>oa_log_frequency</td>
<td>F</td>
<td> +</td>
<td>-</td>
<td>-</td>
<td>+</td>
</tr>
<tr>
<td>oa_log_level</td>
<td>I</td>
<td> +</td>
<td>-</td>
<td>-</td>
<td>+</td>
</tr>
<tr>   <th colspan=9> <a href="#sec:PrimalHeuristics">Primal Heuristics</a> </th>
</tr>
<tr>
<td>feasibility_pump_objective_norm</td>
<td>I</td>
<td> +</td>
<td>+</td>
<td>+</td>
<td>+</td>
</tr>
<tr>
<td>fp_pass_infeasible</td>
<td>S</td>
<td> +</td>
<td>+</td>
<td>+</td>
<td>+</td>
</tr>
<tr>
<td>heuristic_RINS</td>
<td>S</td>
<td> +</td>
<td>+</td>
<td>+</td>
<td>+</td>
</tr>
<tr>
<td>heuristic_dive_MIP_fractional</td>
<td>S</td>
<td> +</td>
<td>+</td>
<td>+</td>
<td>+</td>
</tr>
<tr>
<td>heuristic_dive_MIP_vectorLength</td>
<td>S</td>
<td> +</td>
<td>+</td>
<td>+</td>
<td>+</td>
</tr>
<tr>
<td>heuristic_dive_fractional</td>
<td>S</td>
<td> +</td>
<td>+</td>
<td>+</td>
<td>+</td>
</tr>
<tr>
<td>heuristic_dive_vectorLength</td>
<td>S</td>
<td> +</td>
<td>+</td>
<td>+</td>
<td>+</td>
</tr>
<tr>
<td>heuristic_feasibility_pump</td>
<td>S</td>
<td> +</td>
<td>+</td>
<td>+</td>
<td>+</td>
</tr>
<tr>
<td>pump_for_minlp</td>
<td>S</td>
<td> +</td>
<td>+</td>
<td>+</td>
<td>+</td>
</tr>
<tr>   <th colspan=9> <a href="#sec:Strongbranchingsetup">Strong branching setup</a> </th>
</tr>
<tr>
<td>candidate_sort_criterion</td>
<td>S</td>
<td> +</td>
<td>+</td>
<td>+</td>
<td>+</td>
</tr>
<tr>
<td>maxmin_crit_have_sol</td>
<td>F</td>
<td> +</td>
<td>+</td>
<td>+</td>
<td>+</td>
</tr>
<tr>
<td>maxmin_crit_no_sol</td>
<td>F</td>
<td> +</td>
<td>+</td>
<td>+</td>
<td>+</td>
</tr>
<tr>
<td>min_number_strong_branch</td>
<td>I</td>
<td> +</td>
<td>+</td>
<td>+</td>
<td>+</td>
</tr>
<tr>
<td>number_before_trust_list</td>
<td>I</td>
<td> +</td>
<td>+</td>
<td>+</td>
<td>+</td>
</tr>
<tr>
<td>number_look_ahead</td>
<td>I</td>
<td> +</td>
<td>+</td>
<td>+</td>
<td>+</td>
</tr>
<tr>
<td>number_strong_branch_root</td>
<td>I</td>
<td> +</td>
<td>+</td>
<td>+</td>
<td>+</td>
</tr>
<tr>
<td>setup_pseudo_frac</td>
<td>F</td>
<td> +</td>
<td>+</td>
<td>+</td>
<td>+</td>
</tr>
<tr>
<td>trust_strong_branching_for_pseudo_cost</td>
<td>S</td>
<td> +</td>
<td>+</td>
<td>+</td>
<td>+</td>
</tr>
</tr>
</table>
}
}
\subsection{Algorithm choice}
\label{sec:Algorithmchoice}
\htmlanchor{sec:Algorithmchoice}
\paragraph{algorithm:}\label{opt:algorithm} Choice of the algorithm. \\
 This will preset some of the options of bonmin depending on the algorithm choice. The default value for this string option is "B-BB".
\\ 
Possible values:
\begin{itemize}
   \item B-BB: simple branch-and-bound algorithm,
   \item B-OA: OA Decomposition algorithm,
   \item B-QG: Quesada and Grossmann branch-and-cut algorithm,
   \item B-Hyb: hybrid outer approximation based branch-and-cut,
   \item B-Ecp: ECP cuts based branch-and-cut a la FilMINT.
   \item B-iFP: Iterated Feasibility Pump for MINLP.
\end{itemize}

\subsection{Branch-and-bound options}
\label{sec:Branch-and-boundoptions}
\htmlanchor{sec:Branch-and-boundoptions}
\paragraph{allowable\_fraction\_gap:}\label{opt:allowable_fraction_gap} Specify the value of relative gap under which the algorithm stops. \\
 Stop the tree search when the gap between the objective value of the best known solution and the best bound on the objective of any solution is less than this fraction of the absolute value of the best known solution value. The valid range for this real option is 
$-1 \cdot 10^{+20} \le {\tt allowable\_fraction\_gap } \le 1 \cdot 10^{+20}$
and its default value is $0$.


\paragraph{allowable\_gap:}\label{opt:allowable_gap} Specify the value of absolute gap under which the algorithm stops. \\
 Stop the tree search when the gap between the objective value of the best known solution and the best bound on the objective of any solution is less than this. The valid range for this real option is 
$-1 \cdot 10^{+20} \le {\tt allowable\_gap } \le 1 \cdot 10^{+20}$
and its default value is $0$.


\paragraph{cutoff:}\label{opt:cutoff} Specify cutoff value. \\
 cutoff should be the value of a feasible solution known by the user (if any). The algorithm will only look for solutions better than cutoff. The valid range for this real option is 
$-1 \cdot 10^{+100} \le {\tt cutoff } \le 1 \cdot 10^{+100}$
and its default value is $1 \cdot 10^{+100}$.


\paragraph{cutoff\_decr:}\label{opt:cutoff_decr} Specify cutoff decrement. \\
 Specify the amount by which cutoff is decremented below a new best upper-bound (usually a small positive value but in non-convex problems it may be a negative value). The valid range for this real option is 
$-1 \cdot 10^{+10} \le {\tt cutoff\_decr } \le 1 \cdot 10^{+10}$
and its default value is $1 \cdot 10^{-05}$.


\paragraph{enable\_dynamic\_nlp:}\label{opt:enable_dynamic_nlp} Enable dynamic linear and quadratic rows addition in nlp \\
 The default value for this string option is "no".
\\ 
Possible values:
\begin{itemize}
   \item no
   \item yes
\end{itemize}

\paragraph{integer\_tolerance:}\label{opt:integer_tolerance} Set integer tolerance. \\
 Any number within that value of an integer is considered integer. The valid range for this real option is 
$0 <  {\tt integer\_tolerance } <  {\tt +inf}$
and its default value is $1 \cdot 10^{-06}$.


\paragraph{iteration\_limit:}\label{opt:iteration_limit} Set the cumulative maximum number of iteration in the algorithm used to process nodes continuous relaxations in the branch-and-bound. \\
 value 0 deactivates option. The valid range for this integer option is
$0 \le {\tt iteration\_limit } <  {\tt +inf}$
and its default value is $2147483647$.


\paragraph{nlp\_failure\_behavior:}\label{opt:nlp_failure_behavior} Set the behavior when an NLP or a series of NLP are unsolved by Ipopt (we call unsolved an NLP for which Ipopt is not able to guarantee optimality within the specified tolerances). \\
 If set to "fathom", the algorithm will fathom the node when Ipopt fails to find a solution to the nlp at that node within the specified tolerances. The algorithm then becomes a heuristic, and the user will be warned that the solution might not be optimal. The default value for this string option is "stop".
\\ 
Possible values:
\begin{itemize}
   \item stop: Stop when failure happens.
   \item fathom: Continue when failure happens.
\end{itemize}

\paragraph{node\_comparison:}\label{opt:node_comparison} Choose the node selection strategy. \\
 Choose the strategy for selecting the next node to be processed. The default value for this string option is "best-bound".
\\ 
Possible values:
\begin{itemize}
   \item best-bound: choose node with the smallest bound,
   \item depth-first: Perform depth first search,
   \item breadth-first: Perform breadth first search,
   \item dynamic: Cbc dynamic strategy (starts with a depth first search and turn to best bound after 3 integer feasible solutions have been found).
   \item best-guess: choose node with smallest guessed integer solution
\end{itemize}

\paragraph{node\_limit:}\label{opt:node_limit} Set the maximum number of nodes explored in the branch-and-bound search. \\
 The valid range for this integer option is
$0 \le {\tt node\_limit } <  {\tt +inf}$
and its default value is $2147483647$.


\paragraph{num\_cut\_passes:}\label{opt:num_cut_passes} Set the maximum number of cut passes at regular nodes of the branch-and-cut. \\
 The valid range for this integer option is
$0 \le {\tt num\_cut\_passes } <  {\tt +inf}$
and its default value is $1$.


\paragraph{num\_cut\_passes\_at\_root:}\label{opt:num_cut_passes_at_root} Set the maximum number of cut passes at regular nodes of the branch-and-cut. \\
 The valid range for this integer option is
$0 \le {\tt num\_cut\_passes\_at\_root } <  {\tt +inf}$
and its default value is $20$.


\paragraph{number\_before\_trust:}\label{opt:number_before_trust} Set the number of branches on a variable before its pseudo costs are to be believed in dynamic strong branching. \\
 A value of 0 disables pseudo costs. The valid range for this integer option is
$0 \le {\tt number\_before\_trust } <  {\tt +inf}$
and its default value is $8$.


\paragraph{number\_strong\_branch:}\label{opt:number_strong_branch} Choose the maximum number of variables considered for strong branching. \\
 Set the number of variables on which to do strong branching. The valid range for this integer option is
$0 \le {\tt number\_strong\_branch } <  {\tt +inf}$
and its default value is $20$.


\paragraph{random\_generator\_seed:}\label{opt:random_generator_seed} Set seed for random number generator (a value of -1 sets seeds to time since Epoch). \\
 The valid range for this integer option is
$-1 \le {\tt random\_generator\_seed } <  {\tt +inf}$
and its default value is $0$.


\paragraph{read\_solution\_file:}\label{opt:read_solution_file} Read a file with the optimal solution to test if algorithms cuts it. \\
 For Debugging purposes only. The default value for this string option is "no".
\\ 
Possible values:
\begin{itemize}
   \item no
   \item yes
\end{itemize}

\paragraph{solution\_limit:}\label{opt:solution_limit} Abort after that much integer feasible solution have been found by algorithm \\
 value 0 deactivates option The valid range for this integer option is
$0 \le {\tt solution\_limit } <  {\tt +inf}$
and its default value is $2147483647$.


\paragraph{sos\_constraints:}\label{opt:sos_constraints} Whether or not to activate SOS constraints. \\
 (only type 1 SOS are supported at the moment) The default value for this string option is "enable".
\\ 
Possible values:
\begin{itemize}
   \item enable
   \item disable
\end{itemize}

\paragraph{time\_limit:}\label{opt:time_limit} Set the global maximum computation time (in secs) for the algorithm. \\
 The valid range for this real option is 
$0 \le {\tt time\_limit } <  {\tt +inf}$
and its default value is $1 \cdot 10^{+10}$.


\paragraph{tree\_search\_strategy:}\label{opt:tree_search_strategy} Pick a strategy for traversing the tree \\
 All strategies can be used in conjunction with any of the node comparison functions. Options which affect dfs-dive are max-backtracks-in-dive and max-dive-depth. The dfs-dive won't work in a non-convex problem where objective does not decrease down branches. The default value for this string option is "probed-dive".
\\ 
Possible values:
\begin{itemize}
   \item top-node:  Always pick the top node as sorted by the node comparison function
   \item dive: Dive in the tree if possible, otherwise pick top node as sorted by the tree comparison function.
   \item probed-dive: Dive in the tree exploring two children before continuing the dive at each level.
   \item dfs-dive: Dive in the tree if possible doing a depth first search. Backtrack on leaves or when a prescribed depth is attained or when estimate of best possible integer feasible solution in subtree is worst than cutoff. Once a prescribed limit of backtracks is attained pick top node as sorted by the tree comparison function
   \item dfs-dive-dynamic: Same as dfs-dive but once enough solution are found switch to best-bound and if too many nodes switch to depth-first.
\end{itemize}

\paragraph{variable\_selection:}\label{opt:variable_selection} Chooses variable selection strategy \\
 The default value for this string option is "strong-branching".
\\ 
Possible values:
\begin{itemize}
   \item most-fractional: Choose most fractional variable
   \item strong-branching: Perform strong branching
   \item reliability-branching: Use reliability branching
   \item qp-strong-branching: Perform strong branching with QP approximation
   \item lp-strong-branching: Perform strong branching with LP approximation
   \item nlp-strong-branching: Perform strong branching with NLP approximation
   \item osi-simple: Osi method to do simple branching
   \item osi-strong: Osi method to do strong branching
   \item random: Method to choose branching variable randomly
\end{itemize}

\subsection{ECP cuts generation}
\label{sec:ECPcutsgeneration}
\htmlanchor{sec:ECPcutsgeneration}
\paragraph{ecp\_abs\_tol:}\label{opt:ecp_abs_tol} Set the absolute termination tolerance for ECP rounds. \\
 The valid range for this real option is 
$0 \le {\tt ecp\_abs\_tol } <  {\tt +inf}$
and its default value is $1 \cdot 10^{-06}$.


\paragraph{ecp\_max\_rounds:}\label{opt:ecp_max_rounds} Set the maximal number of rounds of ECP cuts. \\
 The valid range for this integer option is
$0 \le {\tt ecp\_max\_rounds } <  {\tt +inf}$
and its default value is $5$.


\paragraph{ecp\_probability\_factor:}\label{opt:ecp_probability_factor} Factor appearing in formula for skipping ECP cuts. \\
 Choosing -1 disables the skipping. The valid range for this real option is 
${\tt -inf} <  {\tt ecp\_probability\_factor } <  {\tt +inf}$
and its default value is $10$.


\paragraph{ecp\_rel\_tol:}\label{opt:ecp_rel_tol} Set the relative termination tolerance for ECP rounds. \\
 The valid range for this real option is 
$0 \le {\tt ecp\_rel\_tol } <  {\tt +inf}$
and its default value is $0$.


\paragraph{filmint\_ecp\_cuts:}\label{opt:filmint_ecp_cuts} Specify the frequency (in terms of nodes) at which some a la filmint ecp cuts are generated. \\
 A frequency of 0 amounts to to never solve the NLP relaxation. The valid range for this integer option is
$0 \le {\tt filmint\_ecp\_cuts } <  {\tt +inf}$
and its default value is $0$.


\subsection{Feasibility checker using OA cuts}
\label{sec:FeasibilitycheckerusingOAcuts}
\htmlanchor{sec:FeasibilitycheckerusingOAcuts}
\paragraph{feas\_check\_cut\_types:}\label{opt:feas_check_cut_types} Choose the type of cuts generated when an integer feasible solution is found \\
 If it seems too much memory is used should try Benders to use less The default value for this string option is "outer-approx".
\\ 
Possible values:
\begin{itemize}
   \item outer-approx: Generate a set of Outer Approximations cuts.
   \item Benders: Generate a single Benders cut.
\end{itemize}

\paragraph{feas\_check\_discard\_policy:}\label{opt:feas_check_discard_policy} How cuts from feasibility checker are discarded \\
 Normally to avoid cycle cuts from feasibility checker should not be discarded in the node where they are generated. However Cbc sometimes does it if no care is taken which can lead to an infinite loop in Bonmin (usually on simple problems). To avoid this one can instruct Cbc to never discard a cut but if we do that for all cuts it can lead to memory problems. The default policy here is to detect cycles and only then impose to Cbc to keep the cut. The two other alternative are to instruct Cbc to keep all cuts or to just ignore the problem and hope for the best The default value for this string option is "detect-cycles".
\\ 
Possible values:
\begin{itemize}
   \item detect-cycles: Detect if a cycle occurs and only in this case force not to discard.
   \item keep-all: Force cuts from feasibility checker not to be discarded (memory hungry but sometimes better).
   \item treated-as-normal: Cuts from memory checker can be discarded as any other cuts (code may cycle then)
\end{itemize}

\paragraph{generate\_benders\_after\_so\_many\_oa:}\label{opt:generate_benders_after_so_many_oa} Specify that after so many oa cuts have been generated Benders cuts should be generated instead. \\
 It seems that sometimes generating too many oa cuts slows down the optimization compared to Benders due to the size of the LP. With this option we specify that after so many OA cuts have been generated we should switch to Benders cuts. The valid range for this integer option is
$0 \le {\tt generate\_benders\_after\_so\_many\_oa } <  {\tt +inf}$
and its default value is $5000$.


\subsection{MILP Solver}
\label{sec:MILPSolver}
\htmlanchor{sec:MILPSolver}
\paragraph{cpx\_parallel\_strategy:}\label{opt:cpx_parallel_strategy} Strategy of parallel search mode in CPLEX. \\
 -1 = opportunistic, 0 = automatic, 1 = deterministic (refer to CPLEX documentation) The valid range for this integer option is
$-1 \le {\tt cpx\_parallel\_strategy } \le 1$
and its default value is $0$.


\paragraph{milp\_solver:}\label{opt:milp_solver} Choose the subsolver to solve MILP sub-problems in OA decompositions. \\
  To use Cplex, a valid license is required and you should have compiled OsiCpx in COIN-OR  (see Osi documentation). The default value for this string option is "Cbc\_D".
\\ 
Possible values:
\begin{itemize}
   \item Cbc\_D: Coin Branch and Cut with its default
   \item Cbc\_Par: Coin Branch and Cut with passed parameters
   \item Cplex: IBM Cplex
\end{itemize}

\paragraph{milp\_strategy:}\label{opt:milp_strategy} Choose a strategy for MILPs. \\
 The default value for this string option is "solve\_to\_optimality".
\\ 
Possible values:
\begin{itemize}
   \item find\_good\_sol: Stop sub milps when a solution improving the incumbent is found
   \item solve\_to\_optimality: Solve MILPs to optimality
\end{itemize}

\paragraph{number\_cpx\_threads:}\label{opt:number_cpx_threads} Set number of threads to use with cplex. \\
 (refer to CPLEX documentation) The valid range for this integer option is
$0 \le {\tt number\_cpx\_threads } <  {\tt +inf}$
and its default value is $0$.


\subsection{MILP cutting planes in hybrid algorithm}
\label{sec:MILPcuttingplanesinhybridalgorithm}
\htmlanchor{sec:MILPcuttingplanesinhybridalgorithm}
\paragraph{2mir\_cuts:}\label{opt:2mir_cuts} Frequency (in terms of nodes) for generating 2-MIR cuts in branch-and-cut \\
 If $k > 0$, cuts are generated every $k$ nodes, if $-99 < k < 0$ cuts are generated every $-k$ nodes but Cbc may decide to stop generating cuts, if not enough are generated at the root node, if $k=-99$ generate cuts only at the root node, if $k=0$ or $100$ do not generate cuts. The valid range for this integer option is
$-100 \le {\tt 2mir\_cuts } <  {\tt +inf}$
and its default value is $0$.


\paragraph{Gomory\_cuts:}\label{opt:Gomory_cuts} Frequency (in terms of nodes) for generating Gomory cuts in branch-and-cut. \\
 If $k > 0$, cuts are generated every $k$ nodes, if $-99 < k < 0$ cuts are generated every $-k$ nodes but Cbc may decide to stop generating cuts, if not enough are generated at the root node, if $k=-99$ generate cuts only at the root node, if $k=0$ or $100$ do not generate cuts. The valid range for this integer option is
$-100 \le {\tt Gomory\_cuts } <  {\tt +inf}$
and its default value is $-5$.


\paragraph{clique\_cuts:}\label{opt:clique_cuts} Frequency (in terms of nodes) for generating clique cuts in branch-and-cut \\
 If $k > 0$, cuts are generated every $k$ nodes, if $-99 < k < 0$ cuts are generated every $-k$ nodes but Cbc may decide to stop generating cuts, if not enough are generated at the root node, if $k=-99$ generate cuts only at the root node, if $k=0$ or $100$ do not generate cuts. The valid range for this integer option is
$-100 \le {\tt clique\_cuts } <  {\tt +inf}$
and its default value is $-5$.


\paragraph{cover\_cuts:}\label{opt:cover_cuts} Frequency (in terms of nodes) for generating cover cuts in branch-and-cut \\
 If $k > 0$, cuts are generated every $k$ nodes, if $-99 < k < 0$ cuts are generated every $-k$ nodes but Cbc may decide to stop generating cuts, if not enough are generated at the root node, if $k=-99$ generate cuts only at the root node, if $k=0$ or $100$ do not generate cuts. The valid range for this integer option is
$-100 \le {\tt cover\_cuts } <  {\tt +inf}$
and its default value is $0$.


\paragraph{flow\_cover\_cuts:}\label{opt:flow_cover_cuts} Frequency (in terms of nodes) for generating flow cover cuts in branch-and-cut \\
 If $k > 0$, cuts are generated every $k$ nodes, if $-99 < k < 0$ cuts are generated every $-k$ nodes but Cbc may decide to stop generating cuts, if not enough are generated at the root node, if $k=-99$ generate cuts only at the root node, if $k=0$ or $100$ do not generate cuts. The valid range for this integer option is
$-100 \le {\tt flow\_cover\_cuts } <  {\tt +inf}$
and its default value is $-5$.


\paragraph{lift\_and\_project\_cuts:}\label{opt:lift_and_project_cuts} Frequency (in terms of nodes) for generating lift-and-project cuts in branch-and-cut \\
 If $k > 0$, cuts are generated every $k$ nodes, if $-99 < k < 0$ cuts are generated every $-k$ nodes but Cbc may decide to stop generating cuts, if not enough are generated at the root node, if $k=-99$ generate cuts only at the root node, if $k=0$ or $100$ do not generate cuts. The valid range for this integer option is
$-100 \le {\tt lift\_and\_project\_cuts } <  {\tt +inf}$
and its default value is $0$.


\paragraph{mir\_cuts:}\label{opt:mir_cuts} Frequency (in terms of nodes) for generating MIR cuts in branch-and-cut \\
 If $k > 0$, cuts are generated every $k$ nodes, if $-99 < k < 0$ cuts are generated every $-k$ nodes but Cbc may decide to stop generating cuts, if not enough are generated at the root node, if $k=-99$ generate cuts only at the root node, if $k=0$ or $100$ do not generate cuts. The valid range for this integer option is
$-100 \le {\tt mir\_cuts } <  {\tt +inf}$
and its default value is $-5$.


\paragraph{reduce\_and\_split\_cuts:}\label{opt:reduce_and_split_cuts} Frequency (in terms of nodes) for generating reduce-and-split cuts in branch-and-cut \\
 If $k > 0$, cuts are generated every $k$ nodes, if $-99 < k < 0$ cuts are generated every $-k$ nodes but Cbc may decide to stop generating cuts, if not enough are generated at the root node, if $k=-99$ generate cuts only at the root node, if $k=0$ or $100$ do not generate cuts. The valid range for this integer option is
$-100 \le {\tt reduce\_and\_split\_cuts } <  {\tt +inf}$
and its default value is $0$.


\subsection{NLP interface}
\label{sec:NLPinterface}
\htmlanchor{sec:NLPinterface}
\paragraph{nlp\_solver:}\label{opt:nlp_solver} Choice of the solver for local optima of continuous NLP's \\
 Note that option will work only if the specified solver has been installed. Ipopt will usually be installed with Bonmin by default. For FilterSQP please see http://www-unix.mcs.anl.gov/~leyffer/solvers.html on how to obtain it and https://projects.coin-or.org/Bonmin/wiki/HintTricks on how to configure Bonmin to use it. The default value for this string option is "Ipopt".
\\ 
Possible values:
\begin{itemize}
   \item Ipopt: Interior Point OPTimizer (https://projects.coin-or.org/Ipopt)
   \item filterSQP: Sequential quadratic programming trust region algorithm (http://www-unix.mcs.anl.gov/~leyffer/solvers.html)
   \item all: run all available solvers at each node
\end{itemize}

\paragraph{warm\_start:}\label{opt:warm_start} Select the warm start method \\
 This will affect the function getWarmStart(), and as a consequence the warm starting in the various algorithms. The default value for this string option is "none".
\\ 
Possible values:
\begin{itemize}
   \item none: No warm start, just start NLPs from optimal solution of the root relaxation
   \item fake\_basis: builds fake basis, useful for cut management in Cbc (warm start is the same as in none)
   \item optimum: Warm start with direct parent optimum
   \item interior\_point: Warm start with an interior point of direct parent
\end{itemize}

\subsection{NLP solution robustness}
\label{sec:NLPsolutionrobustness}
\htmlanchor{sec:NLPsolutionrobustness}
\paragraph{max\_consecutive\_failures:}\label{opt:max_consecutive_failures} (temporarily removed) Number $n$ of consecutive unsolved problems before aborting a branch of the tree. \\
 When $n > 0$, continue exploring a branch of the tree until $n$ consecutive problems in the branch are unsolved (we call unsolved a problem for which Ipopt can not guarantee optimality within the specified tolerances). The valid range for this integer option is
$0 \le {\tt max\_consecutive\_failures } <  {\tt +inf}$
and its default value is $10$.


\paragraph{max\_random\_point\_radius:}\label{opt:max_random_point_radius} Set max value r for coordinate of a random point. \\
 When picking a random point, coordinate i will be in the interval [min(max(l,-r),u-r), max(min(u,r),l+r)] (where l is the lower bound for the variable and u is its upper bound) The valid range for this real option is 
$0 <  {\tt max\_random\_point\_radius } <  {\tt +inf}$
and its default value is $100000$.


\paragraph{num\_iterations\_suspect:}\label{opt:num_iterations_suspect} Number of iterations over which a node is considered "suspect" (for debugging purposes only, see detailed documentation). \\
 When the number of iterations to solve a node is above this number, the subproblem at this node is considered to be suspect and it will be written into a file (set to -1 to deactivate this). The valid range for this integer option is
$-1 \le {\tt num\_iterations\_suspect } <  {\tt +inf}$
and its default value is $-1$.


\paragraph{num\_retry\_unsolved\_random\_point:}\label{opt:num_retry_unsolved_random_point} Number $k$ of times that the algorithm will try to resolve an unsolved NLP with a random starting point (we call unsolved an NLP for which Ipopt is not able to guarantee optimality within the specified tolerances). \\
 When Ipopt fails to solve a continuous NLP sub-problem, if $k > 0$, the algorithm will try again to solve the failed NLP with $k$ new randomly chosen starting points  or until the problem is solved with success. The valid range for this integer option is
$0 \le {\tt num\_retry\_unsolved\_random\_point } <  {\tt +inf}$
and its default value is $0$.


\paragraph{random\_point\_perturbation\_interval:}\label{opt:random_point_perturbation_interval} Amount by which starting point is perturbed when choosing to pick random point by perturbing starting point \\
 The valid range for this real option is 
$0 <  {\tt random\_point\_perturbation\_interval } <  {\tt +inf}$
and its default value is $1$.


\paragraph{random\_point\_type:}\label{opt:random_point_type} method to choose a random starting point \\
 The default value for this string option is "Jon".
\\ 
Possible values:
\begin{itemize}
   \item Jon: Choose random point uniformly between the bounds
   \item Andreas: perturb the starting point of the problem within a prescribed interval
   \item Claudia: perturb the starting point using the perturbation radius suffix information
\end{itemize}

\paragraph{resolve\_on\_small\_infeasibility:}\label{opt:resolve_on_small_infeasibility} If a locally infeasible problem is infeasible by less than this, resolve it with initial starting point. \\
 It is set to 0 by default with Ipopt. When using FilterSQP, Bonmin sets it to a small value. The valid range for this real option is 
$0 \le {\tt resolve\_on\_small\_infeasibility } <  {\tt +inf}$
and its default value is $0$.


\subsection{NLP solves in hybrid algorithm (B-Hyb)}
\label{sec:NLPsolvesinhybridalgorithm(B-Hyb)}
\htmlanchor{sec:NLPsolvesinhybridalgorithm(B-Hyb)}
\paragraph{nlp\_solve\_frequency:}\label{opt:nlp_solve_frequency} Specify the frequency (in terms of nodes) at which NLP relaxations are solved in B-Hyb. \\
 A frequency of 0 amounts to to never solve the NLP relaxation. The valid range for this integer option is
$0 \le {\tt nlp\_solve\_frequency } <  {\tt +inf}$
and its default value is $10$.


\paragraph{nlp\_solve\_max\_depth:}\label{opt:nlp_solve_max_depth} Set maximum depth in the tree at which NLP relaxations are solved in B-Hyb. \\
 A depth of 0 amounts to to never solve the NLP relaxation. The valid range for this integer option is
$0 \le {\tt nlp\_solve\_max\_depth } <  {\tt +inf}$
and its default value is $10$.


\paragraph{nlp\_solves\_per\_depth:}\label{opt:nlp_solves_per_depth} Set average number of nodes in the tree at which NLP relaxations are solved in B-Hyb for each depth. \\
 The valid range for this real option is 
$0 \le {\tt nlp\_solves\_per\_depth } <  {\tt +inf}$
and its default value is $1 \cdot 10^{+100}$.


\subsection{Nonconvex problems}
\label{sec:Nonconvexproblems}
\htmlanchor{sec:Nonconvexproblems}
\paragraph{coeff\_var\_threshold:}\label{opt:coeff_var_threshold} Coefficient of variation threshold (for dynamic definition of cutoff\_decr). \\
 The valid range for this real option is 
$0 \le {\tt coeff\_var\_threshold } <  {\tt +inf}$
and its default value is $0.1$.


\paragraph{dynamic\_def\_cutoff\_decr:}\label{opt:dynamic_def_cutoff_decr} Do you want to define the parameter cutoff\_decr dynamically? \\
 The default value for this string option is "no".
\\ 
Possible values:
\begin{itemize}
   \item no
   \item yes
\end{itemize}

\paragraph{first\_perc\_for\_cutoff\_decr:}\label{opt:first_perc_for_cutoff_decr} The percentage used when, the coeff of variance is smaller than the threshold, to compute the cutoff\_decr dynamically. \\
 The valid range for this real option is 
${\tt -inf} <  {\tt first\_perc\_for\_cutoff\_decr } <  {\tt +inf}$
and its default value is $-0.02$.


\paragraph{max\_consecutive\_infeasible:}\label{opt:max_consecutive_infeasible} Number of consecutive infeasible subproblems before aborting a branch. \\
 Will continue exploring a branch of the tree until "max\_consecutive\_infeasible"consecutive problems are locally infeasible by the NLP sub-solver. The valid range for this integer option is
$0 \le {\tt max\_consecutive\_infeasible } <  {\tt +inf}$
and its default value is $0$.


\paragraph{num\_resolve\_at\_infeasibles:}\label{opt:num_resolve_at_infeasibles} Number $k$ of tries to resolve an infeasible node (other than the root) of the tree with different starting point. \\
 The algorithm will solve all the infeasible nodes with $k$ different random starting points and will keep the best local optimum found. The valid range for this integer option is
$0 \le {\tt num\_resolve\_at\_infeasibles } <  {\tt +inf}$
and its default value is $0$.


\paragraph{num\_resolve\_at\_node:}\label{opt:num_resolve_at_node} Number $k$ of tries to resolve a node (other than the root) of the tree with different starting point. \\
 The algorithm will solve all the nodes with $k$ different random starting points and will keep the best local optimum found. The valid range for this integer option is
$0 \le {\tt num\_resolve\_at\_node } <  {\tt +inf}$
and its default value is $0$.


\paragraph{num\_resolve\_at\_root:}\label{opt:num_resolve_at_root} Number $k$ of tries to resolve the root node with different starting points. \\
 The algorithm will solve the root node with $k$ random starting points and will keep the best local optimum found. The valid range for this integer option is
$0 \le {\tt num\_resolve\_at\_root } <  {\tt +inf}$
and its default value is $0$.


\paragraph{second\_perc\_for\_cutoff\_decr:}\label{opt:second_perc_for_cutoff_decr} The percentage used when, the coeff of variance is greater than the threshold, to compute the cutoff\_decr dynamically. \\
 The valid range for this real option is 
${\tt -inf} <  {\tt second\_perc\_for\_cutoff\_decr } <  {\tt +inf}$
and its default value is $-0.05$.


\subsection{Outer Approximation Decomposition (B-OA)}
\label{sec:OuterApproximationDecomposition(B-OA)}
\htmlanchor{sec:OuterApproximationDecomposition(B-OA)}
\paragraph{oa\_decomposition:}\label{opt:oa_decomposition} If yes do initial OA decomposition \\
 The default value for this string option is "no".
\\ 
Possible values:
\begin{itemize}
   \item no
   \item yes
\end{itemize}

\subsection{Outer Approximation cuts generation}
\label{sec:OuterApproximationcutsgeneration}
\htmlanchor{sec:OuterApproximationcutsgeneration}
\paragraph{add\_only\_violated\_oa:}\label{opt:add_only_violated_oa} Do we add all OA cuts or only the ones violated by current point? \\
 The default value for this string option is "no".
\\ 
Possible values:
\begin{itemize}
   \item no: Add all cuts
   \item yes: Add only violated cuts
\end{itemize}

\paragraph{oa\_cuts\_scope:}\label{opt:oa_cuts_scope} Specify if OA cuts added are to be set globally or locally valid \\
 The default value for this string option is "global".
\\ 
Possible values:
\begin{itemize}
   \item local: Cuts are treated as locally valid
   \item global: Cuts are treated as globally valid
\end{itemize}

\paragraph{oa\_rhs\_relax:}\label{opt:oa_rhs_relax} Value by which to relax OA cut \\
 RHS of OA constraints will be relaxed by this amount times the absolute value of the initial rhs if it is >= 1 (otherwise by this amount). The valid range for this real option is 
$-0 \le {\tt oa\_rhs\_relax } <  {\tt +inf}$
and its default value is $1 \cdot 10^{-08}$.


\paragraph{tiny\_element:}\label{opt:tiny_element} Value for tiny element in OA cut \\
 We will remove "cleanly" (by relaxing cut) an element lower than this. The valid range for this real option is 
$-0 \le {\tt tiny\_element } <  {\tt +inf}$
and its default value is $1 \cdot 10^{-08}$.


\paragraph{very\_tiny\_element:}\label{opt:very_tiny_element} Value for very tiny element in OA cut \\
 Algorithm will take the risk of neglecting an element lower than this. The valid range for this real option is 
$-0 \le {\tt very\_tiny\_element } <  {\tt +inf}$
and its default value is $1 \cdot 10^{-17}$.


\subsection{Output and Loglevel}
\label{sec:OutputandLoglevel}
\htmlanchor{sec:OutputandLoglevel}
\paragraph{bb\_log\_interval:}\label{opt:bb_log_interval} Interval at which node level output is printed. \\
 Set the interval (in terms of number of nodes) at which a log on node resolutions (consisting of lower and upper bounds) is given. The valid range for this integer option is
$0 \le {\tt bb\_log\_interval } <  {\tt +inf}$
and its default value is $100$.


\paragraph{bb\_log\_level:}\label{opt:bb_log_level} specify main branch-and-bound log level. \\
 Set the level of output of the branch-and-bound : 0 - none, 1 - minimal, 2 - normal low, 3 - normal high The valid range for this integer option is
$0 \le {\tt bb\_log\_level } \le 5$
and its default value is $1$.


\paragraph{file\_solution:}\label{opt:file_solution} Write a file bonmin.sol with the solution \\
 The default value for this string option is "no".
\\ 
Possible values:
\begin{itemize}
   \item yes
   \item no
\end{itemize}

\paragraph{fp\_log\_frequency:}\label{opt:fp_log_frequency} display an update on lower and upper bounds in FP every n seconds \\
 The valid range for this real option is 
$0 <  {\tt fp\_log\_frequency } <  {\tt +inf}$
and its default value is $100$.


\paragraph{fp\_log\_level:}\label{opt:fp_log_level} specify FP iterations log level. \\
 Set the level of output of OA decomposition solver : 0 - none, 1 - normal, 2 - verbose The valid range for this integer option is
$0 \le {\tt fp\_log\_level } \le 2$
and its default value is $1$.


\paragraph{lp\_log\_level:}\label{opt:lp_log_level} specify LP log level. \\
 Set the level of output of the linear programming sub-solver in B-Hyb or B-QG : 0 - none, 1 - minimal, 2 - normal low, 3 - normal high, 4 - verbose The valid range for this integer option is
$0 \le {\tt lp\_log\_level } \le 4$
and its default value is $0$.


\paragraph{milp\_log\_level:}\label{opt:milp_log_level} specify MILP solver log level. \\
 Set the level of output of the MILP subsolver in OA : 0 - none, 1 - minimal, 2 - normal low, 3 - normal high The valid range for this integer option is
$0 \le {\tt milp\_log\_level } \le 4$
and its default value is $0$.


\paragraph{nlp\_log\_at\_root:}\label{opt:nlp_log_at_root} specify a different log level for root relaxation. \\
 The valid range for this integer option is
$0 \le {\tt nlp\_log\_at\_root } \le 12$
and its default value is $0$.


\paragraph{nlp\_log\_level:}\label{opt:nlp_log_level} specify NLP solver interface log level (independent from ipopt print\_level). \\
 Set the level of output of the OsiTMINLPInterface : 0 - none, 1 - normal, 2 - verbose The valid range for this integer option is
$0 \le {\tt nlp\_log\_level } \le 2$
and its default value is $1$.


\paragraph{oa\_cuts\_log\_level:}\label{opt:oa_cuts_log_level} level of log when generating OA cuts. \\
 0: outputs nothing,
1: when a cut is generated, its violation and index of row from which it originates,
2: always output violation of the cut.
3: output generated cuts incidence vectors. The valid range for this integer option is
$0 \le {\tt oa\_cuts\_log\_level } <  {\tt +inf}$
and its default value is $0$.


\paragraph{oa\_log\_frequency:}\label{opt:oa_log_frequency} display an update on lower and upper bounds in OA every n seconds \\
 The valid range for this real option is 
$0 <  {\tt oa\_log\_frequency } <  {\tt +inf}$
and its default value is $100$.


\paragraph{oa\_log\_level:}\label{opt:oa_log_level} specify OA iterations log level. \\
 Set the level of output of OA decomposition solver : 0 - none, 1 - normal, 2 - verbose The valid range for this integer option is
$0 \le {\tt oa\_log\_level } \le 2$
and its default value is $1$.


\subsection{Primal Heuristics}
\label{sec:PrimalHeuristics}
\htmlanchor{sec:PrimalHeuristics}
\paragraph{feasibility\_pump\_objective\_norm:}\label{opt:feasibility_pump_objective_norm} Norm of feasibility pump objective function \\
 The valid range for this integer option is
$1 \le {\tt feasibility\_pump\_objective\_norm } \le 2$
and its default value is $1$.


\paragraph{fp\_pass\_infeasible:}\label{opt:fp_pass_infeasible} Say whether feasibility pump should claim to converge or not \\
 The default value for this string option is "no".
\\ 
Possible values:
\begin{itemize}
   \item no: When master MILP is infeasible just bail out (don't stop all algorithm). This is the option for using in B-Hyb.
   \item yes: Claim convergence, numerically dangerous.
\end{itemize}

\paragraph{heuristic\_RINS:}\label{opt:heuristic_RINS} if yes runs the RINS heuristic \\
 The default value for this string option is "no".
\\ 
Possible values:
\begin{itemize}
   \item no
   \item yes
\end{itemize}

\paragraph{heuristic\_dive\_MIP\_fractional:}\label{opt:heuristic_dive_MIP_fractional} if yes runs the Dive MIP Fractional heuristic \\
 The default value for this string option is "no".
\\ 
Possible values:
\begin{itemize}
   \item no
   \item yes
\end{itemize}

\paragraph{heuristic\_dive\_MIP\_vectorLength:}\label{opt:heuristic_dive_MIP_vectorLength} if yes runs the Dive MIP VectorLength heuristic \\
 The default value for this string option is "no".
\\ 
Possible values:
\begin{itemize}
   \item no
   \item yes
\end{itemize}

\paragraph{heuristic\_dive\_fractional:}\label{opt:heuristic_dive_fractional} if yes runs the Dive Fractional heuristic \\
 The default value for this string option is "no".
\\ 
Possible values:
\begin{itemize}
   \item no
   \item yes
\end{itemize}

\paragraph{heuristic\_dive\_vectorLength:}\label{opt:heuristic_dive_vectorLength} if yes runs the Dive VectorLength heuristic \\
 The default value for this string option is "no".
\\ 
Possible values:
\begin{itemize}
   \item no
   \item yes
\end{itemize}

\paragraph{heuristic\_feasibility\_pump:}\label{opt:heuristic_feasibility_pump} whether the heuristic feasibility pump should be used \\
 The default value for this string option is "no".
\\ 
Possible values:
\begin{itemize}
   \item no
   \item yes
\end{itemize}

\paragraph{pump\_for\_minlp:}\label{opt:pump_for_minlp} whether to run the feasibility pump heuristic for MINLP \\
 The default value for this string option is "no".
\\ 
Possible values:
\begin{itemize}
   \item no
   \item yes
\end{itemize}

\subsection{Strong branching setup}
\label{sec:Strongbranchingsetup}
\htmlanchor{sec:Strongbranchingsetup}
\paragraph{candidate\_sort\_criterion:}\label{opt:candidate_sort_criterion} Choice of the criterion to choose candidates in strong-branching \\
 The default value for this string option is "best-ps-cost".
\\ 
Possible values:
\begin{itemize}
   \item best-ps-cost: Sort by decreasing pseudo-cost
   \item worst-ps-cost: Sort by increasing pseudo-cost
   \item most-fractional: Sort by decreasing integer infeasibility
   \item least-fractional: Sort by increasing integer infeasibility
\end{itemize}

\paragraph{maxmin\_crit\_have\_sol:}\label{opt:maxmin_crit_have_sol} Weight towards minimum in of lower and upper branching estimates when a solution has been found. \\
 The valid range for this real option is 
$0 \le {\tt maxmin\_crit\_have\_sol } \le 1$
and its default value is $0.1$.


\paragraph{maxmin\_crit\_no\_sol:}\label{opt:maxmin_crit_no_sol} Weight towards minimum in of lower and upper branching estimates when no solution has been found yet. \\
 The valid range for this real option is 
$0 \le {\tt maxmin\_crit\_no\_sol } \le 1$
and its default value is $0.7$.


\paragraph{min\_number\_strong\_branch:}\label{opt:min_number_strong_branch} Sets minimum number of variables for strong branching (overriding trust) \\
 The valid range for this integer option is
$0 \le {\tt min\_number\_strong\_branch } <  {\tt +inf}$
and its default value is $0$.


\paragraph{number\_before\_trust\_list:}\label{opt:number_before_trust_list} Set the number of branches on a variable before its pseudo costs are to be believed during setup of strong branching candidate list. \\
 The default value is that of "number\_before\_trust" The valid range for this integer option is
$-1 \le {\tt number\_before\_trust\_list } <  {\tt +inf}$
and its default value is $0$.


\paragraph{number\_look\_ahead:}\label{opt:number_look_ahead} Sets limit of look-ahead strong-branching trials \\
 The valid range for this integer option is
$0 \le {\tt number\_look\_ahead } <  {\tt +inf}$
and its default value is $0$.


\paragraph{number\_strong\_branch\_root:}\label{opt:number_strong_branch_root} Maximum number of variables considered for strong branching in root node. \\
 The valid range for this integer option is
$0 \le {\tt number\_strong\_branch\_root } <  {\tt +inf}$
and its default value is $2147483647$.


\paragraph{setup\_pseudo\_frac:}\label{opt:setup_pseudo_frac} Proportion of strong branching list that has to be taken from most-integer-infeasible list. \\
 The valid range for this real option is 
$0 \le {\tt setup\_pseudo\_frac } \le 1$
and its default value is $0.5$.


\paragraph{trust\_strong\_branching\_for\_pseudo\_cost:}\label{opt:trust_strong_branching_for_pseudo_cost} Whether or not to trust strong branching results for updating pseudo costs. \\
 The default value for this string option is "yes".
\\ 
Possible values:
\begin{itemize}
   \item no
   \item yes
\end{itemize}

