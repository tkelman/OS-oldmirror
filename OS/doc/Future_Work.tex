% This is a list of future work. As a stand-alone file it does not process in TeX or LaTeX,
% but it can be \input into the osUsersManual, and it is small enough so that it can be
% mainained by a simple text editor.
%
% This version dated 5 February 2010
%
%%%%%%%%%%%%%%%%%%%%%%%%%%%%%%%%%%%%%%%%%%%%%%%%%%%%%%%%%%%%%%%%%%%%%%%%%%%%%%%%%%%%%%%%%%%%

\begin{itemize}
\item Implement a Gurobi solver interface.

\item Implement the {\tt mult} and {\tt incr} features in OSInstance/OSiL parsers.

\item Implement the SOS feature in OSiL

\item Implement a switch so that the solver output is put into OSrL output. This output should go into a {\tt solutionResult} element in {\tt otherSolutionResults.}

\item Put the GAMS OSiL read and OSrL write into the OSModelingInterfaces

\item Implement the Bcp solver

\item Implement OS as part of CoinUtils. (That is, break out some of the basic routines.)

\item Write a document on how to hook your solver to OS

\item Add a module to FlopC++ that writes OSiL

\item Add an OS option to the OSSolverInterfaces that allows the user to get the log file of the solver. The user would have to use the specific solver option to set the level of log output.

\item Investigate the Amazon cloud computing

\item Installer for Windows

\item Treat \url{https://projects.coin-or.org/OS/ticket/14} 

\item Figure out how to put the version number on the executables

\item Add code so that we can take a LINGO postfix problem instance and generate an OSExpression tree

\item Right now the OSSolverService command line parser requires / for path -- allow {$\backslash$} for Windows users

\item Prepare constraint programming document/report

\item Prepare a paper on OSOption and OSResult.

\item Write documentation on the new Java example 

\item Build and document Java distribution for users who want to build OSiL from Java and 
call OSSolverService from Java. 

\item Warmstart for LP

\item Paper on SOS

\item Vet/finalize SDPA and verify SDPA2OSiL

\item Implement parser for matrices and cones

\item Proof of Concept: Hook to a solver (CSDP?)

\item Paper on matrix programming

\item Re-check scenario formulation

\item Implement parser for scenarios

\item Implement solver (DE/decomposition)

\item Put in proper error checking for problems that have zero variables

\item Put in a detailed example of how to build a problem instance using the OSInstance API

\item Run Artistic Style on the code so Gus and Kipp are consistent

\item Solution files for matrix programming (Imre)

\item Complex numbers in OSiL (Imre)

\item Figure out why Che-lin's problem dies when finding a sparse Hessian

\item GAMS list -- SOS can be used with Cbc and Bonmin, semicontinuous+semiinteger
variables with Cbc, user defined functions with Ipopt and Bonmin,
parameters for Cbc, and - most important - you can redirect the output.

\end{itemize}
