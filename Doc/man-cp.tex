%===========================================================================%
%                                                                           %
% This file is part of the documentation for the SYMPHONY MILP Solver.      %
%                                                                           %
% SYMPHONY was jointly developed by Ted Ralphs (ted@lehigh.edu) and         %
% Laci Ladanyi (ladanyi@us.ibm.com).                                        %
%                                                                           %
% (c) Copyright 2000-2013 Ted Ralphs. All Rights Reserved.                  %
%                                                                           %
% SYMPHONY is licensed under the Eclipse Public License. Please see         %
% accompanying file for terms.                                              %
%                                                                           %
%===========================================================================%

%%%%%%%%%%%%%%%%%%%%%%%%%%%%%%%%%%%%%%%%%%%%%%%%%%%%%%%%%%%%%%%%%%%%%%%%%%%%%
\subsection{Cut pool module callbacks}

Due to the relative simplicity of the cut pool, there are no wrapper
functions implemented for CP. Consequently, there are no default
options and no post-processing.

%begin{latexonly}
\bd
%end{latexonly}

%%%%%%%%%%%%%%%%%%%%%%%%%%%%%%%%%%%%%%%%%%%%%%%%%%%%%%%%%%%%%%%%%%%%%%%%%%%%%
% user_receive_cutpool_data
%%%%%%%%%%%%%%%%%%%%%%%%%%%%%%%%%%%%%%%%%%%%%%%%%%%%%%%%%%%%%%%%%%%%%%%%%%%%%

\firstfuncdef{user\_receive\_cp\_data}
\label{user_receive_cp_data}
\sindex[cf]{user\_receive\_cp\_data}
\begin{verbatim}
int user_receive_cp_data(void **user)
\end{verbatim}

\bd

\item[Description:] \hfill

The user has to receive here all problem-specific information sent
from \hyperref{{\tt user\_send\_cp\_data()}}{\ptt{user\_send\_cp\_data()}
(see Section }{)}{user_send_cp_data} function in the master module.
The user has to allocate space for all the data structures, including
{\tt user} itself. Note that this function is only called if the
either the Tree Manager, LP, or CP are running as a separate process
(i.e.~either {\tt COMPILE\_IN\_TM}, {\tt COMPILE\_IN\_LP}, or {\tt
COMPILE\_IN\_CP} are set to {\tt FALSE} in the {\tt make} file).
Otherwise, this is done in {\tt \htmlref{user\_send\_cp\_data()}
{user_send_cp_data}}. See the
description of that function for more details.

\item[Arguments:] \hfill

\bt{llp{280pt}}
{\tt void **user} & INOUT & Pointer to the user-defined data structure. \\
\et

\returns

\bt{lp{300pt}}
{\tt USER\_ERROR} & Error. Cut pool module exits. \\
{\tt USER\_SUCCESS} & The user received data successfully. \\
{\tt USER\_DEFAULT} & The user did not send any data to be received. \\
\et

\item[Invoked from:] {\tt cp\_initialize} at module start-up.

\ed

\vspace{1ex}

%%%%%%%%%%%%%%%%%%%%%%%%%%%%%%%%%%%%%%%%%%%%%%%%%%%%%%%%%%%%%%%%%%%%%%%%%%%%%
% user_receive_lp_solution_cp
%%%%%%%%%%%%%%%%%%%%%%%%%%%%%%%%%%%%%%%%%%%%%%%%%%%%%%%%%%%%%%%%%%%%%%%%%%%%%

\functiondef{user\_receive\_lp\_solution\_cp}
\sindex[cf]{user\_receive\_lp\_solution\_cp}
\begin{verbatim}
void user_receive_lp_solution_cp(void *user)
\end{verbatim}

\bd

\item[Description:] \hfill

This function is invoked only in the case parallel computation and only if in
the {\tt \htmlref{user\_send\_lp\_solution()}{user_send_lp_solution}} function
of the LP module, the user opted for packing the current LP solution in a
custom format. Here she must receive the data she sent there.

\item[Arguments:] \hfill

\bt{llp{250pt}}
{\tt void *user} & IN & Pointer to the user-defined data structure. \\
\et

\returns

\bt{lp{300pt}}
{\tt USER\_ERROR} & Cuts are not checked for this LP solution. \\
{\tt USER\_SUCCESS} & The user function executed properly. \\
{\tt USER\_DEFAULT} & SYMPHONY's default format should be used. \\
\et

\item[Invoked from:] Whenever an LP solution is received.

\item[Note:] \hfill

\BB\ automatically unpacks the level, index and iteration number
corresponding to the current LP solution within the current search
tree node.

\ed

\vspace{1ex}

%%%%%%%%%%%%%%%%%%%%%%%%%%%%%%%%%%%%%%%%%%%%%%%%%%%%%%%%%%%%%%%%%%%%%%%%%%%%%
% user_prepare_to_check_cuts
%%%%%%%%%%%%%%%%%%%%%%%%%%%%%%%%%%%%%%%%%%%%%%%%%%%%%%%%%%%%%%%%%%%%%%%%%%%%%

\functiondef{user\_prepare\_to\_check\_cuts}
\sindex[cf]{user\_prepare\_to\_check\_cuts}
\label{user_prepare_to_check_cuts}
\begin{verbatim}
int user_prepare_to_check_cuts(void *user, int varnum, int *indices, 
                               double *values)
\end{verbatim}

\bd

\item[Description:] \hfill

This function is invoked after an LP solution is received but before any cuts
are tested. Here the user can build up data structures
(e.g., a graph representation of the solution) that can make the testing of
cuts easier in the {\tt \htmlref{user\_check\_cuts}{user_check_cuts}} function.

\item[Arguments:] \hfill

\bt{llp{285pt}}
{\tt void *user} & IN & Pointer to the user-defined data structure. \\
{\tt int varnum} & IN & The number of nonzero/fractional variables described
in {\tt indices} and {\tt values}. \\
{\tt int *indices} & IN & The user indices of the nonzero/fractional
variables. \\
{\tt double *values} & IN & The nonzero/fractional values. \\ 
\et

\returns

\bt{lp{300pt}}
{\tt USER\_ERROR} & Cuts are not checked for this LP solution. \\
{\tt USER\_SUCCESS} & The user is prepared to check cuts. \\
{\tt USER\_DEFAULT} & There are no user-defined cuts in the pool. \\
\et

\item[Invoked from:] Whenever an LP solution is received.

\ed

%%%%%%%%%%%%%%%%%%%%%%%%%%%%%%%%%%%%%%%%%%%%%%%%%%%%%%%%%%%%%%%%%%%%%%%%%%%%%
% user_check_cut
%%%%%%%%%%%%%%%%%%%%%%%%%%%%%%%%%%%%%%%%%%%%%%%%%%%%%%%%%%%%%%%%%%%%%%%%%%%%%

\functiondef{user\_check\_cut}
\label{user_check_cuts}
\sindex[cf]{user\_check\_cut}
\begin{verbatim}
int user_check_cut(void *user, double lpetol, int varnum, 
                   int *indices, double *values, cut_data *cut,
                   int *is_violated, double *quality)
\end{verbatim}

\bd

\item[Description:] \hfill

The user has to determine whether a given cut is violated by the given
LP solution (see Section \ref{user-written-lp} for a description of
the {\tt \htmlref{cut\_data data}{cut_data}} data structure). Also,
the user can assign a number to the cut called the {\em quality}. This
number is used in deciding which cuts to check and purge. See the
section on \htmlref{Cut Pool Parameters}{cut_pool_params} for more
information.

\item[Arguments:] \hfill

\bt{llp{255pt}}
{\tt void *user} & INOUT & The user defined part of p. \\
{\tt double lpetol} & IN & The $\epsilon$ tolerance in the LP module. \\
{\tt int varnum} & IN & Same as the previous function. \\
{\tt int *indices} & IN & Same as the previous function. \\
{\tt double *values} & IN & Same as the previous function. \\
{\tt cut\_data *cut} & IN & Pointer to the cut to be tested. \\
{\tt int *is\_violated} & OUT & TRUE/FALSE based on whether the cut is violated
or not. \\
{\tt double *quality} & OUT & a number representing the relative
strength of the cut.
\et

\returns

\bt{lp{300pt}}
{\tt USER\_ERROR} & Cut is not sent to the LP, regardless of the value of
{\tt *is\_violated}. \\
{\tt USER\_SUCCESS} & The user function exited properly. \\
{\tt USER\_DEFAULT} & Same as error. \\
\et

\item[Invoked from:] Whenever a cut needs to be checked.

\item[Note:] \hfill

The same note applies to {\tt number}, {\tt indices} and {\tt values} as in
the previous function.

\ed

\vspace{1ex}

%%%%%%%%%%%%%%%%%%%%%%%%%%%%%%%%%%%%%%%%%%%%%%%%%%%%%%%%%%%%%%%%%%%%%%%%%%%%%
% user_finished_to_check_cuts
%%%%%%%%%%%%%%%%%%%%%%%%%%%%%%%%%%%%%%%%%%%%%%%%%%%%%%%%%%%%%%%%%%%%%%%%%%%%%

\functiondef{user\_finished\_checking\_cuts}
\sindex[cf]{user\_finished\_checking\_cuts}
\begin{verbatim}
int user_finished_checking_cuts(void *user)
\end{verbatim}

\bd

\item[Description:] \hfill

When this function is invoked there are no more cuts to be checked, so the
user can dismantle data structures he created in {\tt 
\htmlref{user\_prepare\_to\_check\_cuts}{user_prepare_to_check_cuts}}. Also, 
if he received and stored the LP solution himself he can delete it now.

\item[Arguments:] \hfill

\bt{llp{250pt}}
{\tt void *user} & IN & Pointer to the user-defined data structure. \\
\et

\returns

\bt{lp{300pt}}
{\tt USER\_ERROR} & Ignored. \\
{\tt USER\_SUCCESS} & The user function exited properly. \\
{\tt USER\_DEFAULT} & There are no user-defined cuts in the pool. \\
\et

\item[Invoked from:] After all cuts have been checked.

\ed

\vspace{1ex}

%%%%%%%%%%%%%%%%%%%%%%%%%%%%%%%%%%%%%%%%%%%%%%%%%%%%%%%%%%%%%%%%%%%%%%%%%%%%%
% user_free_cp
%%%%%%%%%%%%%%%%%%%%%%%%%%%%%%%%%%%%%%%%%%%%%%%%%%%%%%%%%%%%%%%%%%%%%%%%%%%%%

\functiondef{user\_free\_cp}
\sindex[cf]{user\_free\_cp}
\begin{verbatim}
int user_free_cp(void **user)
\end{verbatim}

\bd

\item[Description:] \hfill

The user has to free all the data structures within {\tt user}, and also free
{\tt user} itself. The user can use the built-in macro {\tt FREE} that checks
the existence of a pointer before freeing it. 

\item[Arguments:] \hfill

\bt{llp{280pt}}
{\tt void **user} & INOUT & Pointer to the user-defined data structure
(should be {\tt NULL} on exit). \\
\et

\returns

\bt{lp{300pt}}
{\tt USER\_ERROR} & Ignored. \\
{\tt USER\_SUCCESS} & The user freed all data structures. \\
{\tt USER\_DEFAULT} & There is nothing to be freed. \\
\et

\item[Invoked from:] cp\_close() at module shutdown.

\ed

\vspace{1ex}

%begin{latexonly}
\ed
%end{latexonly}
